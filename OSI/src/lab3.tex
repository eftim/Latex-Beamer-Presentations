\documentclass[12pt,a4paper]{exam}
\usepackage{amsmath}
\usepackage{amsfonts}
\usepackage{amssymb}
\usepackage[T2A]{fontenc}
\usepackage[utf8]{inputenc}
\usepackage{listings}
\usepackage{color}

\lstset{language=C,captionpos=b,
tabsize=4,frame=lines,
basicstyle=\ttfamily,
keywordstyle=\color{blue},
commentstyle=\color{lightgray},
stringstyle=\color{red},
breaklines=true,showstringspaces=false}
\begin{document}
\pagestyle{headandfoot}
\header{\textbf{ФИНКИ\\Основи на софтверско
инженерство}}{}{\large{\textbf{Лабораториска вежба 3}}}
\headrule
\cfoot{Страна \thepage}
\begin{center}
\Large{\textbf{Работа со системи за контрола на верзии - SVN}}
\end{center}

\section{Инсталирајте SVN client}

\begin{enumerate}
  \item Симнете го од некоја од следните локации:\\
  http://files.finki.ukim.mk/uploads/tomche.delev/RapidSVNPortable.exe\\ 
  http://rapidshare.com/files/28919598/RapidSVNPortable.exe
  \item Отакако ќе се симне изврешете го и отпакувајте го во "My Documents"
  \item Иврешете ја апликацијата "PortableRapidSVN.exe"
\end{enumerate}
 
\section{Повлекување на работната копија (SVN checkout)}
\begin{enumerate}
  \item Од менито на апликацијата изберете \textbf{Repository->Checkout...} (Ctrl+O)
  \item Во полето *URL* внесете ја следната патека\\
  \linebreak
  http://svn.finki.ukim.mk/svn/OSI2011/
  \linebreak
  \item Во полето \textbf{Destination Directory} изберете патека до директориумот во
  кој ќе се смести вашата работна копија од репозиторито (нека биде некој
  фолдер во MyDocuments пример ``osisvn'')
  \item Кликнете ОК и почекајте да се симне работната копија 
  \item Како корисничко име и лозинка внесете некоја комбинација од корисничките
  имиња и лозинки на крајот на овој документ
 \end{enumerate}

\texttt{svn checkout}

\section{Додавање содржина во работната копија}
\textbf{Ова го работите во синхронизација со вашиот партнер во вежбата
колегата/колешката лево/десно од вас (по договор)}
\begin{enumerate}
  \item Преку ``Windows Explorer'' креирајте нов фолдер и именувајте го
 ``indeks1\_indeks2'' каде што ``indeks1'' и ``indeks2'' се вашиот и индексот на колегата со кој работите заедно (пр. 111500\_111600).
 \textbf{Ова го прави само едниот од партнерите во вежбата.}
 \item Во овој фолдер додадете ги задачите од лабораториската вежба КРС (само .c
 датотеките). Едниот од вас нека ги додаде првите 2 задачи, потоа другиот партнер ќе ги додаде останатите.
\end{enumerate}
  
\texttt{svn add}
  
\section{Исраќање на содржината во репозиториумот (SVN commit)}

\begin{enumerate}
  \item Со десен клик во главниот прозорец на вашиот директориум ``indeks1\_indeks2''
 изберете \textbf{Add recursive} (ова го додава директориумот и се што се содржи во него)
  \item Со десен клик на вашиот директориум изберете \textbf{Commit...}.
  \item По ова вашата содржина е додадена во репозиториумот.
\end{enumerate}

\texttt{svn commit}

\section{Повлекување на последната содржина од репозитори (SVN update)}

\begin{enumerate}
  \item Со десен клик на вашиот директориум изберете \textbf{Update...} (доколку сакате
  да ги повлечете промените од цела работна копија во која работат сите ваши
  колеги изврешете ја оваа наредба на почетниот директориум)
  \item По извршување на оваа наредба вие и вашиот партнер за вежбата треба да
  имате иста содржина во вашите работни копии на вашиот заеднички директориум
\end{enumerate}

\texttt{svn update}

\section{Менување на содржината на одредена датотека}
\begin{enumerate}
  \item Променете ја содржината на накоја од датотеките и испратете ги
  промените. Потоа повлечете ја најновата верзија и видете ги промените кои ги
  направил вашиот партнер.
\end{enumerate}
   
\section{Бришење на датотеки}

\begin{enumerate}
  \item Преку \textbf{Windows explorer} избрешете некоја датотека во вашиот
  работен директориум. Потоа повлечете ја најновата верзија од репозиториумот
  (SVN update)
  \item Што се случува со избришаната датотека?
  \item За да избришете датотека од вашиот работен директориум, а потоа во слениот
  \textbf{Commit} да се избрише и од репозиториумот треба преку работниот
  прозороц на RapidSVN од десен клик на датаотеката која сакате да ја избришете
  треба да изберете \textbf{Delete}. Со извршување на \textbf{Commit} по оваа акција
  датотеката ќе биде избришана од репозиториумот.
\end{enumerate}
 
\texttt{svn delete}\\

\emph{Сите останати акции за управување со датотеки и директориуми како
преместување \textbf{(Cut\&Paste)} или преименување \textbf{(Rename)} мора да се прават преку работниот
прозорец на \textbf{RapidSVN (Move, Rename)}}

\section{Менување датотеки (Конфликти)}

 Двајцата партнери во вежбата направете промена во иста датоека заменувајќи ја истата линија со нешто по ваш избор
  Испратете ги промените со SVN commit и справете се со конфликтот кој ќе
 настане

\section{Експериментирајте со другите наредби}

  Пробајте ги останатите команди во RapidSVN како *Log*, *Info* итн.

\section{Кориснички имиња и лозинки}
Изберете некоја од следните кориснички имиња и лозинки. Изборот нека биде според
следната формула:
\begin{verbatim}
indeks % 20 + 1 (Pr. 111205 % 20 + 1 = 6) osiUser06 8270
Username    Password
osiUser02   6908
osiUser03   7761
osiUser04   5572
osiUser05   3319
osiUser06   8270
osiUser07   6757
osiUser08   8509
osiUser09   2360
osiUser10   6533
osiUser11   7938
osiUser12   2799
osiUser13   5000
osiUser14   7366
osiUser15   3040
osiUser16   9085
osiUser17   1446
osiUser18   8209
osiUser19   5791
osiUser20   9635
\end{verbatim}


\end{document}