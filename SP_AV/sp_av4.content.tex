% DO NOT COMPILE THIS FILE DIRECTLY!
% This is included by the other .tex files.
\section{циклуси}
\begin{frame}[fragile]{Задачa 1а}
Да се напише програма за пресметување на сумата на сите парни двоцифрени броеви. Добиената сума се печати на екран.
\pause
\begin{exampleblock}{Решение}
    \begin{lstlisting}
#include <stdio.h> 
int main () { 
    int i = 10, suma = 0; 
    while (i <= 98) { 
        suma = suma + i; 
        i+=2; 
    } 
    printf("Sumata na site parni dvocifreni broevi e %d\n", suma); 
    return 0; 
}
\end{lstlisting}
\end{exampleblock}
\end{frame}


\begin{frame}[fragile]{Задачa 1б}
\begin{scriptsize}
Да се напише програма за пресметување на сумата на сите непарни двоцифрени броеви. Програмата ја печати сумата на екран во следниот формат: \texttt{11 + 13 + 15 + 17 + ... + 97 + 99 = 2475}\\
\textbf{Забелешка: Програмата да се направи без користење на наредбата if}
\end{scriptsize}
\pause
\begin{columns}
\column{.5\textwidth}
\begin{exampleblock}{Решение - Верзија 1}
\begin{lstlisting}
#include <stdio.h>
int main () { 
    int i = 11, suma = 0; 
    printf("%d", i); 
    suma = i; 
    i=i+2; 
    while (i <= 99){ 
        printf(" + %d", i); 
        suma = suma + i; 
        i+=2; 
    } 
    printf(" = %d\n", suma); 
    return 0; 
}
\end{lstlisting}
\end{exampleblock}
\pause
\column{.5\textwidth}
\begin{exampleblock}{Решение - Верзија 2}
\begin{lstlisting}
#include <stdio.h> 
int main () { 
    int i = 11, suma = 0; 
    while (i <= 97) { 
        printf("%d + ", i); 
        suma = suma + i; 
        i+=2; 
    } 
    printf(" %d", i); 
    suma = suma + i; 
    printf(" = %d\n", suma); 
    return 0; 
} 
\end{lstlisting}
\end{exampleblock}
\end{columns}
\end{frame}

\begin{frame}[fragile]{Задачa 2}{Решение со употреба на \texttt{while} и
\texttt{do\ldots while}} Да се напише програма за пресметување на $y = x^n$ за
даден природен број $n, n>=1$ и реален број $x$.
\pause
\begin{columns}
\column{.5\textwidth}
\begin{exampleblock}{Решение - со употреба на \texttt{while}}
\begin{lstlisting}
#include <stdio.h> 
int main () { 
    int brojac = 0, n; 
    float x, y = 1; 
    printf("vnesi ja osnovata: "); 
    scanf("%f", &x); 
    printf("vnesi go eksponentot: "); 
    scanf("%d", &n);    
    while (brojac < n) { 
        y *= x; 
        brojac++; 
    } 
    printf("%f^%d = %f\n", x, n, y); 
    return 0; 
}
\end{lstlisting}
\end{exampleblock}
\pause
\column{.5\textwidth}
\begin{exampleblock}{Решение - со употреба на \texttt{do...while}}
\begin{lstlisting}
#include <stdio.h> 
int main () { 
    int brojac = 0, n; 
    float x, y = 1; 
    printf("vnesi ja osnovata: "); 
    scanf("%f", &x); 
    printf("vnesi go eksponentot: "); 
    scanf("%d", &n); 
    do { 
        y *= x; 
        brojac++; 
    } while (brojac < n); 
    printf("%f^%d = %f\n", x, n, y); 
    return 0; 
 }
\end{lstlisting}
\end{exampleblock}
\end{columns}
\end{frame}

\begin{frame}[fragile]{Задачa 2}{Решение со употреба на \texttt{for}}
Да се напише програма за пресметување на $y = x^n$ за даден природен број $n,
n>=1$ и реален број $x$.
\pause
\begin{exampleblock}{Решение со употреба на \texttt{for}}
\begin{lstlisting}
#include <stdio.h> 
int main () { 
    int brojac = 0, n; 
    float x, y = 1; 
    printf("vnesi ja osnovata: "); 
    scanf("%f", &x); 
    printf("vnesi go eksponentot: "); 
    scanf("%d", &n); 
    for(brojac = 1, y = x; brojac < n; brojac++) {
    //for(; brojac < n; brojac++) {
        x *= y;
    } 
    printf("%f^%d = %f\n", x, n, y); 
    return 0; 
 }
\end{lstlisting}
\end{exampleblock}
\end{frame}


\begin{frame}[fragile]{Задачa 3}
Да се напише програма која од n броеви (внесени од тастатура) ќе го определи бројот на броеви што се деливи со 3, при делењето со 3 имаат остаток 1, односно 2.\\
\textbf{Забелeшка: Задачата да се реши со while, do…while и for}
\end{frame}


\begin{frame}[fragile]{Решение на задачa 3}{\texttt{while}}
\begin{exampleblock}{Решение на задачата со употреба на \texttt{while}}
\begin{lstlisting}
#include <stdio.h> 
int main () { 
    int n = 1, i = 0, broj, del, os1, os2; 
    del = os1 = os2 = 0; 
    printf("Kolku broevi treba da se proveruvaat za delivost so 3?\n"); 
    scanf("%d", &n); 
    while (i < n) { 
        printf("Vnesete broj za proverka: "); 
        scanf("%d", &broj); 
        if (broj % 3 == 0) 
            del++; 
        else if (broj % 3 == 1) 
            os1++;
        else os2++; 
        i++; 
    }
    printf("%d broj(a) se delivi so 3.\n", del); 
    printf("%d broj(a) imaat ostatok 1, pri delenje so 3.\n", os1); 
    printf("%d broj(a) imaat ostatok 2, pri delenje so 3.\n", os2); 
    return 0; 
} 
\end{lstlisting}
\end{exampleblock}
\end{frame}


\begin{frame}[fragile]{Решение на задачa 3}{\texttt{do while}}
\begin{exampleblock}{Решение на задачата со употреба на \texttt{do\ldots while}}
\begin{lstlisting}
#include <stdio.h>
int main () { 
    int n = 1, i = 0, broj, del, os1, os2; 
    del = os1 = os2 = 0;
    printf("Kolku broevi treba da se proveruvaat za delivost so 3?\n"); 
    scanf("%d", &n); 
    do { 
        printf("Vnesete broj za proverka: "); 
        scanf("%d", &broj); 
        if (broj % 3 == 0) 
            del++; 
        else if (broj % 3 == 1) 
            os1++; 
        else 
            os2++; 
        i++; 
    } while (i < n); 
    printf("%d broj(a) se delivi so 3.\n", del); 
    printf("%d broj(a) imaat ostatok 1, pri delenje so 3.\n", os1); 
    printf("%d broj(a) imaat ostatok 2, pri delenje so 3.\n", os2); 
    return 0; 
}
\end{lstlisting}
\end{exampleblock}
\end{frame}


\begin{frame}[fragile]{Решение на задачa 3}{\texttt{for}}
\begin{exampleblock}{Решение на задачата со употреба на \texttt{for}}
\begin{lstlisting}
#include <stdio.h> 
int main () { 
    int n = 1, i = 0, broj, del, os1, os2; 
    del = os1 = os2 = 0; 
    printf("Kolku broevi treba da se proveruvaat za delivost so 3?\n"); 
    scanf("%d", &n); 
    for (i = 0; i < n;i++) { 
        printf("Vnesete broj za proverka: "); 
        scanf("%d", &broj); 
        if (broj % 3 == 0) 
            del++; 
        else if ( broj % 3 == 1) 
            os1++; 
        else 
            os2++; 
    }
    printf("%d broj(a) se delivi so 3.\n", del); 
    printf("%d broj(a) imaat ostatok 1, pri delenje so 3.\n", os1); 
    printf("%d broj(a) imaat ostatok 2, pri delenje so 3.\n", os2); 
    return 0; 
}
\end{lstlisting}
\end{exampleblock}
\end{frame}

\begin{frame}[fragile]{Задачa 4}
Да се напише програма која која на екран ќе ги испечати сите четири-цифрени броеви кај кои збирот на трите 
најмалку значајни цифри е еднаков со најзначајната цифра.
\begin{exampleblock}{Пример}
    \texttt{4031 (4=0+3+1), 5131 (5=1+3+1)}
\end{exampleblock}
\end{frame}

\begin{frame}[fragile]{Решение на задачa 4}
\begin{exampleblock}{Решение}
    \begin{lstlisting}
#include <stdio.h> 
int main() { 
    int m, i, n, suma, prva_cifra, cifra; 
    i = 1000; 
    while (i<=9999) { 
        prva_cifra = i/1000; 
        n = i % 1000; 
        suma = 0; 
        while (n > 0) { 
            cifra = n % 10; 
            suma += cifra; 
            n /= 10; 
        } 
        if (suma == prva_cifra) printf("%d\t", i); 
        i++; 
    } 
    return 0; 
}
    \end{lstlisting}
\end{exampleblock}
\end{frame}

\begin{frame}[fragile]{Задачa 5}
Да се напише програма која ќе ги испечати сите броеви од зададен опсег кои исто
се читаат и одлево надесно и оддесно налево.
\begin{exampleblock}{Пример}
\texttt{12345    54321}
\end{exampleblock}
\end{frame}

\begin{frame}[fragile]{Решение 5}
\begin{exampleblock}{Решение}
\begin{lstlisting}
#include <stdio.h> 
int main () { 
    int i, odb, dob, pom, prev, cifra; 
    printf("Vnesete vrednost za opsegot.\n"); 
    printf("Od koj broj?\n"); scanf("%d", &odb); 
    printf("Do koj broj?\n"); scanf("%d", &dob); 
    for (i = odb; i <= dob; i++) { 
        pom = i; 
        prev = 0; 
        while (pom > 0) { 
            cifra = pom % 10; 
            prev = prev*10 + cifra; 
            pom /= 10; 
        } 
        if (prev == i) printf("%d\t", i); 
    } 
    return 0; 
}
\end{lstlisting}
\end{exampleblock}
\end{frame}


\begin{frame}[fragile]{Задачa 6}
\scriptsize{Да се напише програма која од непознат број на цели броеви кои се внесуваат 
од тастатура ќе го определи бројот со максимална вредност. 
Програмата завршува ако наместо број се внесе знак што не е цифра.}
\pause
\begin{exampleblock}{Решение}
\begin{lstlisting}
#include <stdio.h> 
int main() { 
    int broj, max; 
    if (scanf("%d", &max)){ 
        while(scanf("%d", &broj)){ 
            if(max < broj){ 
                max = broj; 
            } 
        } 
        printf("Maksimalniot broj e %d", max); 
    } else { 
        printf("Treba da vnesete najmalku eden cel broj"); 
    } 
    return 0; 
}
\end{lstlisting}
\end{exampleblock}
\end{frame}


\begin{frame}[fragile]{Задачa 7}
Да се напише програма која од непознат број на цели броеви кои се внесуваат од 
тастатура ќе го определи бројот со максимална вредност. 
Притоа броевите поголеми од 100 не се земаат предвид т.е. се игнорираат. 
Програмата завршува ако наместо број се внесе знак што не е цифра.
\end{frame}

\begin{frame}[fragile]{Решение на задача 7}
\begin{exampleblock}{Решение}
\begin{lstlisting}
#include <stdio.h> 
int main() { 
    int broj, max; 
   if (scanf("%d", &max)) { 
        while(scanf("%d", &broj)) { 
            if (broj > 100) continue; 
            if(max < broj){ 
                max = broj; 
            } 
        } 
        printf("Maksimalniot broj e %d", max); 
    } else { 
        printf("Treba da vnesete najmalku eden cel broj"); 
    } 
    return 0; 
}
\end{lstlisting}
\end{exampleblock}
\end{frame}

\begin{frame}{Задачa 8}
Да се напише програма која од непознат број на цели броеви кои се внесуваат од тастатура 
ќе ги определи двата броја со најголеми вредности. Програмата завршува ако наместо 
број се внесе знак што не е цифра. 
\begin{exampleblock}{Пример}
Ако се внесат броевите \texttt{2 4 7 4 2 1 8 6 9 7 10 3} програмата ќе отпечати 
\texttt{10} и \texttt{9}.
\end{exampleblock}
\end{frame}

\begin{frame}[fragile]{Решение на задача 8}
\begin{exampleblock}{Решение}
\begin{lstlisting}
#include <stdio.h> 
int main() { 
    int broj, max1, max2, pom; 
    if (scanf("%d%d", &max1, &max2) == 2) { 
        if (max2>max1){ 
            pom = max1; 
            max1 = max2; 
            max2 = pom; 
        } 
        while(scanf("%d", &broj)) { 
            if(broj > max1){ 
                max2 = max1; 
                max1 = broj; 
            } else if (broj>max2) { 
                max2 = broj; 
            } 
        } 
        printf("Brojot so najgolema vrednost e %d\n", max1); 
        printf("Brojot so vtora najgolema vrednost e %d\n", max2); 
    } else { 
        printf("Treba da vnesete najmalku dva celi broja"); 
    } 
    return 0; 
}
\end{lstlisting}
\end{exampleblock}
\end{frame}

\begin{frame}{Задачa 9}
Да се напише програма која од N цели броеви внесени од тастатура ќе ја определи 
разликата од сумите на броевите на парни и непарни позиции 
(според редоследот на внесување). Ако оваа разлика е помала од 10 на екран се печати 
"Dvete sumi se slicni" а во спротивно на екран се печати "Dvete sumi mnogu se
razlikuvaat".
\begin{exampleblock}{Пример}
За броевите внесени од тастатура:\\
\texttt{{\color{red}2} 4 {\color{red}3} 4 {\color{red}2} 1 {\color{red}1} 6
{\color{red} 1} 7}\\
\texttt{{\color{red} suma\_neparni\_pozicii = 9}}\\
\texttt{suma\_parni\_pozicii = 22}\\
На екран ќе се испечати: \texttt{Dvete sumi mnogu se razlikuvaat}
\end{exampleblock}
\end{frame}

\begin{frame}[fragile]{Решение на задача 9}
\begin{exampleblock}{Решение}
\begin{lstlisting}
#include <stdio.h> 
//#include <math.h> 
int main() { 
    int razlika, i, n = 0, broj = 0; 
    int suma_neparni_pozicii = 0, suma_parni_pozicii = 0; 
    scanf("%d", &n); 
    for (i = 1; i <= n; i++){ 
        scanf("%d", &broj); 
        if (i % 2){ 
            suma_neparni_pozicii += broj; 
        } else { 
            suma_parni_pozicii += broj; 
        } 
    } 
    razlika = suma_parni_pozicii - suma_neparni_pozicii; 
    //printf("razlikata e %d", razlika); 
    //if(abs(razlika) < 10){ 
    if(razlika < 10 && razlika > -10){ 
        printf("Dvete sumi se slicni"); 
    } else { 
        printf("Dvete sumi mnogu se ralikuvaat"); 
    } 
    return 0; 
}
\end{lstlisting}
\end{exampleblock}
\end{frame}

\begin{frame}{Задачa 10}
Да се напише програма која од непознат број на цели броеви кои се внесуваат од тастатура 
ќе ги определи позициите (редните броеви на внесување) на двата последователни броеви кои 
ја имаат најголемата сума. Програмата завршува ако едно по друго (последователно) 
се внесат два негативни цели броја.
\end{frame}

\begin{frame}[fragile]{Решение на задача 10}
\begin{exampleblock}{Решение}
\begin{lstlisting}
#include <stdio.h> 
int main() { 
    int pol_pozicija, pozicija, max_suma, suma, prethoden, sleden; 
    scanf("%d%d", &prethoden, &sleden); 
    pol_pozicija = pozicija = 2; 
    max_suma = suma = prethoden + sleden; 
    while(1){ 
        if (prethoden < 0 && sleden < 0){ 
            break; 
        } 
        suma = prethoden + sleden; 
        if (suma > max_suma){ 
            max_suma = suma; 
            pol_pozicija = pozicija; 
        } 
        prethoden = sleden; 
        scanf("%d", &sleden); 
        pozicija++; 
    }
    if(pozicija > 2) 
        printf("broevite se naogaat na pozicija %d i %d a nivanata suma e %d",
    pol_pozicija - 1, pol_pozicija, max_suma); 
    return 0; 
}
\end{lstlisting}
\end{exampleblock}
\end{frame}

\section{наредба \texttt{switch}}
\begin{frame}{Задачa 1}
Да се напише програма што ќе овозможи претворање на двоцифрените броеви во
зборови на следниот начин:\\
За двоцифрениот борј 89 на екран ќе се испечати
"osum devet".
\end{frame}

\begin{frame}[t,fragile,shrink=35]{Решение на задача 1}
\begin{columns}
\column{.5\textwidth}
\begin{exampleblock}{Решение прв дел}
\begin{lstlisting}
#include <stdio.h>
int main() {
    int broj, mala, golema;
    printf("Vnesete dvocifren broj:");
    scanf("%d", &broj);
    mala = broj % 10;
    golema = broj/10;
    switch (golema) {
        case 0:
            printf("nula ");
            break;
        case 1:
            printf("eden ");
            break;
        case 2:
            printf("dva ");
            break;
        case 3:
            printf("tri ");
            break;
        case 4:
            printf("cetiri ");
            break;
        case 5:
            printf("pet ");
            break;
        case 6:
            printf("sest ");
            break;
        case 7:
            printf("sedum ");
            break;
        case 8:
            printf("osum ");
            break;
        case 9:
            printf("devet ");
            break;
        default:
            break;
    }
\end{lstlisting}
\end{exampleblock}
\column{.5\textwidth}
\begin{exampleblock}{Решение втор дел}
\begin{lstlisting}
    switch (mala) {
        case 0:
            printf("nula\n");
            break;
        case 1:
            printf("eden\n");
            break;
        case 2:
            printf("dva\n");
            break;
        case 3:
            printf("tri\n");
            break;
        case 4:
            printf("cetiri\n");
            break;
        case 5:
            printf("pet\n");
            break;
        case 6:
            printf("sest\n");
            break;
        case 7:
            printf("sedum\n");
            break;
        case 8:
            printf("osum\n");
            break;
        case 9:
            printf("devet\n");
            break;
        default:
            break;
    }
    printf("%d %d\n", golema, mala);
    return (0);
}
\end{lstlisting}
\end{exampleblock}
\end{columns}
\end{frame}


\begin{frame}{Задачa 2}
Да се напише програма која ќе претставува едноставен калкулатор. Во програмата
се вчитуваат два броја и оператор во формат:\\
\texttt{broj1 operator broj2}\\
По извршената операција во зависност од операторот, се печати резултатот во
формат:\\
\texttt{broj1 operator broj2 = rezultat}
\end{frame}

\begin{frame}[fragile,shrink=10]{Решение на задача 2}
\begin{exampleblock}{Решение}
\begin{lstlisting}
#include <stdio.h>
int main() {
    char op;
    float br1, br2, rez = 0;
    printf("Vnesete dva broja i operator vo format:\n");
    printf("broj1 operator broj2\n");
    scanf("%f %c %f",&br1, &op, &br2);
    switch (op) {
        case '+':
            rez = br1 + br2;
            break;
        case '-':
            rez = br1 - br2;
            break;
        case '*':
            rez = br1 * br2;
            break;
        case '/':
            if (br2 == 0) {
                printf("Greshka: Delenje so 0\n");
                printf(" operacijata ke se ignorira\n");
            }
            else {
                rez = br1 / br2;
            }
            break;
        default:
            printf("Nepoznat operator %c\n", op);
        break;
    }
    if(res) printf("Rezultatot od operacijata: %.2f %c %.2f = %f", br1, op, br2,
    rez); return (0);
}
\end{lstlisting}
\end{exampleblock}
\end{frame}

\begin{frame}{Материјали}{}
    Предавања, аудиториски вежби, соопштенија\\
    \href{http://courses.finki.ukim.mk/}{\textbf{courses.finki.ukim.mk}}
    \vfill
    Изворен код на сите примери и задачи\\
    \href{http://bitbucket.org/tdelev/finki-sp/}{\textbf{bitbucket.org/tdelev/finki-sp}}
    \vfill
    {\Huge Прашања ?}
\end{frame}
