\begin{frame}[fragile]{Задачa 1}
Да се напише програма во која од тастатура се внесува матрица со димензии MxN (M
и N не се поголеми од 100). Програмата треба да ја трансформира матрицата на тој
начин што од секој број ќе го одземе просекот (средната вредност) на
редицата во која припаѓа тој број.
\begin{exampleblock}{Пример}
\[
 \begin{matrix}
  4 & 2 & 7 & 11 & (6) \\
  3 & 8 & 16 & 1 & (7) \\
  17 & 8 & 9 & 5 & (9.75) \\
  6 & 14 & 4 & 7 & (7.75)
 \end{matrix}
 \Longrightarrow
 \begin{matrix}
 -2 & -4 & 1 & 5\\
 -4 & 1 & 9 & -6\\
 7.25 & -1.75 & -0.75 & -4.75\\
 1.75 & 6.25 & -3.75 & -0.75
 \end{matrix}
\]
\end{exampleblock}
\end{frame}

\begin{frame}[fragile]{Задачa 1}{Решение}
\begin{columns}
\column{.5\textwidth}
    \begin{lstlisting}
#include <stdio.h>
#define MAX 100
int main() {
    int a[MAX][MAX], M, N, i, j;
    int suma[MAX];
    printf("Vnesete M i N: \n");
    scanf("%d %d", &M, &N);
    printf("Vnesete ja matricata: \n");
    for (i = 0; i < M; i++) {
        suma[i] = 0;
        for (j = 0; j < N; j++) {
            printf("a[%d][%d] = ", i, j);
            scanf("%d", &a[i][j]);
        }
    }
    for (i = 0; i < M; i++) {
        for (j = 0; j < N; j++) {
            suma[i] += a[i][j];
        }
        for (j = 0; j < N; j++) {
            a[i][j] -= suma[i] * 1.0 / N;
        }
    }
\end{lstlisting}    
    \column{.5\textwidth}
\begin{lstlisting}
    printf("Rezultantnata matrica e: \n");
    for (i = 0; i < M; i++) {
        for (j = 0; j < N; j++) {
            printf("%d\t", a[i][j]);
        }
        printf("\n");
    }
    return 0;
}
\end{lstlisting}
\end{columns}
\end{frame}

\begin{frame}{Задачa 2}
Да се напише програма во која од тастатура се внесува матрица со димензии M и
N. Да се пресмета збирот на сите елементи чии што збир на соседи по хоризонтала
е поголем од збирот на соседите по вертикала на тој елемент. Максимална големина
на матриците е 100 x 100.
\begin{exampleblock}{Пример}
\[
 \begin{matrix}
  4 & \color{red}{2} & 7 & 11 \\
  \color{blue}{3} & \textbf{8} & \color{blue}{16} & 1 \\
  17 & \color{red}{8} & 9 & 5 \\
  6 & 14 & 4 & 7
 \end{matrix}
\]
\end{exampleblock}
\end{frame}

\begin{frame}[fragile,shrink=5]{Задача 2}{Решение}
\begin{lstlisting}
#include <stdio.h>
#define MAX 100
int main() {
    int a[MAX][MAX], M, N, i, j;
    int suma = 0;
    printf("Vnesete M i N: \n");
    scanf("%d %d", &M, &N);
    printf("Vnesete ja matricata: \n");
    for (i = 0; i < M; i++) {
        for (j = 0; j < N; j++) {
            printf("a[%d][%d] = ", i, j);
            scanf("%d", &a[i][j]);
        }
    }
    for (i = 0; i < M; i++) {
        for (j = 0; j < N; j++) {
            int sh = 0;
            int sv = 0;
            if(j > 0) sh += a[i][j - 1];
            if(j < N - 1) sh += a[i][j + 1];
            if(i > 0) sv += a[i - 1][j];
            if(i < M - 1) sv += a[i + 1][j];
            if(sh > sv) suma += a[i][j];
        }
    }
    printf("Sumata e: %d\n", suma);
    return 0;
}
\end{lstlisting}
\end{frame}

\begin{frame}{Задачa 3}
Да се напише програма во која се внесува квадратна матрица од цели броеви со
непарен број на редици и колони. Матрицата да се измени на таков начин што
елементите од главната и споредната дијагонала симетрично ќе се пресликаат во
однос на централниот елемент на матрицата. На крај да се отпечати променетата
матрица.
\begin{exampleblock}{Пример}
\[
 \begin{matrix}
 \color{red}{3} & 4 & 5 & 6 & 7\\ 
 1 & \color{red}{2} & 3 & 6 & 4\\
 4 & 2 & \color{green}{7} & 9 & 1\\
 1 & 9 & 0 & \color{blue}{3} & 5\\ 
 4 & 6 & 2 & 8 & \color{blue}{1}
 \end{matrix}
 \Longrightarrow
 \begin{matrix}
    \color{blue}{1} & 4 & 5 & 6 & 4\\ 
    1 & \color{blue}{3} & 3 & 9 & 4\\
    4 & 2 & \color{green}{7} & 9 & 1\\
    1 & 6 & 0 & \color{red}{2} & 5\\ 
    7 & 6 & 2 & 8 & \color{red}{3}
 \end{matrix}
\]
\end{exampleblock}
\end{frame}

\begin{frame}[fragile,shrink=5]{Задача 3}{Решение}
\begin{lstlisting}
#include <stdio.h>
#define MAX 100
int main() {
    int a[MAX][MAX], M, N, i, j;
    printf("Vnesete M i N: \n");
    scanf("%d %d", &M, &N);
    printf("Vnesete ja matricata: \n");
    for (i = 0; i < M; i++) {
        for (j = 0; j < N; j++) {
            printf("a[%d][%d] = ", i, j);
            scanf("%d", &a[i][j]);
        }
    }
    for (i = 0; i < M / 2; i++) {
        int temp = a[i][i];
        a[i][i] = a[M - 1 - i][M - 1 - i];
        a[M - 1 - i][M - 1 - i] = temp;
        temp = a[i][M - 1 - i];
        a[i][M - 1 - i] = a[M - 1 - i][i];
        a[M - 1 - i][i] = temp;
    }
    printf("Rezultantnata matrica e: \n");
    for (i = 0; i < M; i++) {
        for (j = 0; j < N; j++) {
            printf("%d\t", a[i][j]);
        }
        printf("\n");
    }
    return 0;
}
\end{lstlisting}
\end{frame}

\begin{frame}{Материјали}{}
	Предавања, аудиториски вежби, соопштенија\\
	\href{http://courses.finki.ukim.mk/}{\textbf{courses.finki.ukim.mk}}
	\vfill
	Изворен код на сите примери и задачи\\
	\href{http://bitbucket.org/tdelev/finki-sp/}{\textbf{bitbucket.org/tdelev/finki-sp}}
	\vfill
	{\Huge Прашања ?}
\end{frame}
