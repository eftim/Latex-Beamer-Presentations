% DO NOT COMPILE THIS FILE DIRECTLY!
% This is included by the other .tex files.

\section{Функции}
\begin{frame}[fragile]{Задачa 1}
Пример за \texttt{void} функции и функции без параметри.
\pause
\begin{exampleblock}{Решение}
    \begin{lstlisting}
#include <stdio.h>
/* Deklaracija na funkcii */
void printMax(int broj);
void printPozdrav();
int main() {
    int k = 10;
    printPozdrav(); /* Pecati Pozdrav */
    printMax(k); /* Ja pecati maximalnata vrednost */
    return 0;
}
/* Definicija na funkciite */
void printMax(int broj) {
    printf("Maksimalniot broj e %d\n",broj);
}
void printPozdrav() {
    printf("Dobar Den. Kako se cuvstvuvate denes?\n");
}
\end{lstlisting}
\end{exampleblock}
\end{frame}


\begin{frame}[fragile]{Задачa 2}
Да се напише програма која ќе ги отпечати сите четирицифрени природни броеви кои се деливи со збирот на двата броја 
составен од првите две цифри и од последните две цифри на четирицифрениот број,
и на крајот ќе отпечати колку вакви броеви се пронајдени.\\
\begin{exampleblock}{Пример}
\texttt{3417 е делив со 34 + 17, 5265, 6578, ....}\\
\end{exampleblock}
\end{frame}

\begin{frame}[fragile]{Задача 2}{Решение}
\begin{lstlisting}
#include <stdio.h>
int zb2cif(int n);
int main() {
    int br=0,i;
    for (i=1000; i<=9999; i++) {
        if (i%zb2cif(i)==0) {
            printf("Brojot %d go zadovoluva uslovot\n",i);
            br++;
        }
    }
    printf("Pronajdeni se %d broevi koi go zadovoluvaat uslovot\n",br);
    return 0;
}

int zb2cif(int n) {
    int zbir;
    zbir=(n%100)+(n/100);
    return zbir;
}
\end{lstlisting}
\end{frame}

\begin{frame}{Задачa 3}
Да се напише програма која за даден природен број ја пресметува разликата меѓу најблискиот поголем од него прост број и тој број.
\end{frame}

\begin{frame}[fragile]{Задачa 3}{Решение}
\begin{lstlisting}
#include <stdio.h>
int prost(int n);
int prostgore(int n);
int main() {
    int broj,razlika;
    printf("Vnesi broj\n");
    scanf("%d",&broj);
    razlika=prostgore(broj)-broj;
    printf("Razlikata medu prostiot broj %d i %d e %d\n",prostgore(broj),broj,razlika);
    return 0;
}
int prost(int n) {
    int k;
    k=2;
    while (k*k<=n) {
        if (n%k==0) return 0;
        k++;
    }
    return 1;
}
int prostgore(int n) {
    do
        n++;
    while (!(prost(n)));
    return n;
}
\end{lstlisting}
\end{frame}

\begin{frame}{Задачa 4}
Да се напише програма што ќе ги отпечати сите прости броеви помали од 10000 чиј што збир на цифри е исто така прост број. 
На крајот да се отпечати колку вакви броеви биле пронајдени. 
\texttt{пример: 23, 179, 9613, ...}\\
\end{frame}


\begin{frame}[fragile,shrink=10]{Задачa 4}{Решение}
\begin{lstlisting}
#include <stdio.h>
int eprost(int n);
int zbircif(int n);
int main () {
    int br=0,i;
    for (i=2; i<=9999; i++) {
        if (eprost(i) && eprost(zbircif(i))) {
            printf("Brojot %d go zadovoluva uslovot\n",i);
            br++;
        }
    }
    printf("Pronajdeni se %d broevi koi go zadovoluvaat uslovot\n",br);
    return 0;
}
int eprost(int n) {
    int i, prost;
    if (n<4) prost=1;
    else
    if ((n%2)==0) prost =0;
    else {
        i=3; prost=1;
        while ((i*i<=n) && prost) {
            if (n%i==0) prost=0;
            i+=2;
        }
    }
    return prost;
}
int zbircif(int n) {
    int zbir=0;
    while (n>0) {
        zbir+=(n%10);
        n/=10;
    }
    return zbir;
}
\end{lstlisting}
\end{frame}


\begin{frame}{Задачa 5}
Да се напише програма што ќе ги отпечати сите парови прости броеви што се разликуваат меѓу себе за 2. 
На крај да се отпечати и нивниот број.
\end{frame}


\begin{frame}[fragile,shrink=10]{Задачa 5}{Решение}
\begin{lstlisting}
#include <stdio.h>
int eprost(int n);
int main () {
    int br=0,i;
    for (i=1; i<=(1000-2); i++) {
        if (eprost(i) && eprost(i+2)) {
            printf("Prostire broevi %d I %d se razlikuvaat za 2\n", i, (i+2));
            br++;
        }
    }
    printf("Pronajdeni se vkupno %d parovi prosti broevi koi go zadovoluvaat uslovot\n",br);
    return 0;
}
int eprost(int n) {
    int i;
    if (n < 4) return 1;
    else
    if ((n%2)==0) return 0;
    else {
        i=3;
        while (i*i<=n){
            if (n%i==0) return 0;
            i+=2;
        }
    }
    return 1;
} 
\end{lstlisting}
\end{frame}


\begin{frame}{Задачa 6}
Да се напише функција што прима два параметра $x$ и $n$ и враќа:
\[
   f(n) = \left\{ 
  \begin{array}{l l}
    x + \frac{x^n}{n} + \frac{x^{n+2}}{n + 2} & ,x\geq0\\
    - \frac{x^{n - 1}}{n - 1} + \frac{x^{n+1}}{n + 1} & ,x<0\\
  \end{array} \right.
\]
Потоа да се состави програма што ќе ја табелира оваа функција за прочитано n во
интервал $x\in[-4, 4]$, со чекор $0.1$.
\end{frame}

\begin{frame}[fragile,shrink=10]{Задачa 6}{Решение}
\begin{lstlisting}
#include <stdio.h>
double f(float i, int j);
float stepen(float i, int j);
int main () {
    int n;
    float x;
    printf("Vnesi broj:\n");
    scanf("%d", &n);
    if ((n>=-2) && (n<=1))
        printf("Neodredeno.\n");
    else {
        x=-4.0;
        while (x<=4) {
            printf("x=%3.1f, f(x)=%10.4f\n", x, f(x,n));
            x+=0.1;
        }
    }
    return 0;
}
double f(float i,int j) {
    double vrednost;
    if (i>0)
        vrednost=i+stepen(i,j)-stepen(i,j+2);
    else
        vrednost=-stepen(i,j-1)+stepen(i,j+1);
    return vrednost;
}
float stepen(float i,int j) {
    int k;
    double vrednost;
    if (i==0)
        vrednost=0.0;
    else {
        vrednost=1.0;
        for (k=1;k<=j;++k)
            vrednost*=i;
    }
    return vrednost;
}
\end{lstlisting}
\end{frame}

\section{Рекурзија}
\begin{frame}[fragile]{Задачa 7}
Да се напише програма која пресметува факториел на даден број. Факториел да
се пресметува во посебна рекурзивна функција.
\[
   n! = n * (n - 1) * (n - 2) \ldots * 2 * 1
\]
\pause
\begin{lstlisting}
#include <stdio.h>
int factorial(int n) {
    if(n == 0) return 1;
    return n * factorial(n - 1);
}
int main () {
    int n;
    printf("Vnesi broj:\n");
    scanf("%d", &n);
    printf("%d! = %d\n", n, factorial(n));
    return 0;
}
\end{lstlisting}
\end{frame}

\begin{frame}[fragile]{Задачa 8}
Да се напише програма која ја пресметува сумата на следната низа:
\[
    \begin{array}{l}
    1 + \frac{1}{2} + \frac{1}{3} + \ldots + \frac{1}{n}
    \end{array}
\]
\pause
\begin{lstlisting}
#include <stdio.h>
int sum(int n) {
    if(n == 1) return 1;
    return 1.0 / n + sum(n - 1);
}
int main () {
    int n;
    printf("Vnesi broj:\n");
    scanf("%d", &n);
    printf("sum(%d) = %.2f\n", n, sum(n));
    return 0;
}
\end{lstlisting}
\end{frame}



\begin{frame}{Задачa 9}
Да се напише програма која за дадено N ќе го испише соодветниот Фибоначиев
број. Фибоначиевите броеви се дефинирани на следниов начин:
\[
   \begin{array}{l}
   a_1 = 1\\
   a_2 = 1\\ 
   \vdots\\
   a_n = a_{n - 1} + a_{n - 2}
   \end{array}
\]
\begin{exampleblock}{Пример}
\texttt{1, 1, 2, 3, 5, 8, 13, 21, 34, \ldots} 
\end{exampleblock}
\end{frame}


\begin{frame}[fragile]{Задачa 9}{Решение}
\begin{lstlisting}
#include <stdio.h>
int fibonaci(int n) {
    if(n == 0 || n == 1) return 1;
    return fibonaci(n - 1) + fibonaci(n - 2);
}
int main () {
    int n;
    printf("Vnesi broj:\n");
    scanf("%d", &n);
    printf("fib(%d) = %d\n", n, fibonaci(n));
    return 0;
}
\end{lstlisting}
\end{frame}

\begin{frame}{Задачa 10}
Да се напише програма што ќе ја испишува вредноста на произволен член на
низата дефинирана со:
\[
   \begin{array}{l}
   x_1 = 1\\
   x_2 = 2\\ 
   \vdots\\
   x_n = \frac{n - 1}{n}x_{n - 1} + \frac{1}{n}x_{n - 2}
   \end{array}
\]
\end{frame}


\begin{frame}[fragile]{Задачa 10}{Решение}
\begin{lstlisting}
#include <stdio.h>
float xnn(float x1, float x2, int n) {
    if(n == 1) return 1;
    if(n == 2) return 2;
    return (n - 1) * xnn(x1, x2, n - 1) / n + xnn(x1, x2, n - 2) / n;
}
int main () {
    int n;
    printf("Vnesi n:\n");
    scanf("%d", &n);
    printf("xnn(1, 1, %d) = %.2f\n", n, xnn(1, 1, n));
    return 0;
}
\end{lstlisting}
\end{frame}

\begin{frame}{Материјали}{}
    Предавања, аудиториски вежби, соопштенија\\
    \href{http://courses.finki.ukim.mk/}{\textbf{courses.finki.ukim.mk}}
    \vfill
    Изворен код на сите примери и задачи\\
    \href{http://bitbucket.org/tdelev/finki-sp/}{\textbf{bitbucket.org/tdelev/finki-sp}}
    \vfill
    {\Huge Прашања ?}
\end{frame}
