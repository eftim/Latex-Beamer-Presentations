\documentclass[12pt,a4paper]{exam}
\usepackage{amsmath}
\usepackage{amsfonts}
\usepackage{amssymb}
\usepackage[T2A]{fontenc}
\usepackage[utf8]{inputenc}
\usepackage{listings}
\usepackage{color}

\lstset{language=C,captionpos=b,
tabsize=4,frame=lines,
basicstyle=\ttfamily,
keywordstyle=\color{blue},
commentstyle=\color{lightgray},
stringstyle=\color{red},
breaklines=true,showstringspaces=false}

\begin{document}
\pagestyle{headandfoot}
\header{\textbf{ФИНКИ\\Структурирано
програмирање}}{}{\large{\textbf{Лабораториска вежба 6}}}
\headrule
\cfoot{Страна \thepage}
\begin{center}
\Large{\textbf{Низи}}
\end{center}
\begin{questions}

\question
За внесена низа од N (максимум 100) елементи да се пресмета  и отпечати
просекот на првата и на втората половина на низата. Втората половина на низата
започнува од индекс $i = N / 2$.
\\\emph{На пример}:\\
\texttt{1 5 6 8 11 3 6}\\ 
Prva polovina: \texttt{4}\\
Vtora polovina: \texttt{7}

\question
За внесена низа од N (максимум 100) елементи кои претставуваат висини на патека
за пешачење да се пресмета вкупното искачување/симнување. Притоа искачувањето се
множи со 2. 
\\\emph{На пример}:\\
\texttt{10 14 8 6 15}\\
\texttt{2 * 4 + 6 + 2 + 2 * 9 = 34}

\question
Внесена низа од N (максимум 100) елементи да се трансформира на тој начин што
секоја 0 ќе се замени со најголемиот непарен елемент десно од нулата во
остатокот од низата. Ако не постои непарен елемент во остатокот од низата нулата
си останува.\\\emph{На пример}:\\
\texttt{3 0 4 11 0 5 0 2}\\
\texttt{2 11 4 11 5 5 0 2}

\question
За внесен природен број N пополнете низа со N * N елементи на следниот начин. За
внесено N = 3 се добива \texttt{0 0 1, 0 2 1, 3 2 1} (Запирките се ставени за
логички да ги одвојат групите на елементи). \\\emph{Пример за 2}:\\
\texttt{0 1 2 1}

\end{questions}
\end{document}