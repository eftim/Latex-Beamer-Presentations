\documentclass[12pt,a4paper]{exam}
\usepackage{amsmath}
\usepackage{amsfonts}
\usepackage{amssymb}
\usepackage[T2A]{fontenc}
\usepackage[utf8]{inputenc}
\usepackage{listings}
\usepackage{color}

\lstset{language=C,captionpos=b,
tabsize=4,frame=lines,
basicstyle=\ttfamily,
keywordstyle=\color{blue},
commentstyle=\color{lightgray},
stringstyle=\color{red},
breaklines=true,showstringspaces=false}

\begin{document}
\pagestyle{headandfoot}
\header{\textbf{ФИНКИ\\Структурирано програмирање}}{}{\large{\textbf{Лабораториска вежба 1}}}
\headrule
\cfoot{Страна \thepage}
\begin{center}
\Large{\textbf{Променливи, операции, влез и излез на податоци}}
\end{center}
\begin{questions}
\question
Креирање на прв проект во CodeBlocks
\begin{enumerate}
\item Овторете ја апликацијата \textbf{Code::Blocks IDE}
\item Креирајте нов проект \textbf{(File->New->Project...) / (Create new project)}.
\item Од понудените категории изберете \textbf{"Empty project"}.
\item Именувајте го проектот како \textbf{"KRS{indeks}-{lab}-{zadaca}"} Пример. KRS113001-1-1 и сместете го во директориумот "My Documents\textbackslash KRS" кој ако не е креиран го креирате.
\item На следниот поглед за опцијата компајлер треба да ви е избран "GNU GCC Compiler" и да се избрани опциите за "Create debug configuration" и "Create release configuration" (Ова треба да се стандардните опции).
\item Во вака креираниот проект додадете нова датотека \textbf{File->New file...} Изберете C/C++ source тип на датотека и изберете \textbf{C} како програмски јазик.
\item На следниот поглед кликнете на \textbf{...} и внесете име на изворната датотека. Потоа треба да е избрано "Add file to active project In build target(s)" \textbf{Debug} и \textbf{Release} (All->Finish).
\end{enumerate}
\question
Пишување и извршување на првата C програма
\begin{enumerate}
\item Во текстуалниот уредувач внесете го следниот изворен код:
\begin{lstlisting}
#include <stdio.h>

int main() {
    int a, b;
    printf("a = ");
    scanf("%d", &a);
    printf("b = ");
    scanf("%d", &b);
    printf("a + b = %d\n", a + b);
    printf("a - b = %d\n", a - b);
    printf("a * b = %d\n", a * b);
    printf("a / b = %d\n", a / b);
    printf("a %% b = %d\n", a % b);
    return 0;
}
\end{lstlisting}
\item Извршете ја програмата со избор на \textbf{"Build->Build and run" (F9)}
\item Внесете вредности во конзолата и тестирајте ја програмата.
\end{enumerate}

\question
Креирајте втор проект според упатството од претходната задача (KRSbr\_indeks12) и внесете го следниот изворен код за пресметување на индекс на телесна маса. Извршете ја и тестирајте ја програмата за вашите податоци.
\begin{lstlisting}
#include <stdio.h>
int main() {
    float masa, visina;
    printf("Vnesete visina(cm): ");
    scanf("%f", &visina);
    printf("Vnesete tezhina(kg): ");
    scanf("%f", &masa);
    visina /= 100;
    float bmi = masa / (visina * visina);
    printf("Vasiot BMI e: %.2f\n", bmi);
    return 0;
}
\end{lstlisting}

\question
Да се напише програма која за даден цел број секунди кој се внесува од тастатура, ќе ги отпечати на екран соодветните вредности во часови, минути и секунди.
\\На пример:\\
\texttt{7555}\\
\texttt{7555 sekundi se 2 casovi, 5 minuti i 55 sekundi}

\question
Напишете ја следната програма:
\begin{lstlisting}
#include <stdio.h>
int main() {
    int d; char c;
    printf("Vnesete eden broj i eden znak: ");
    scanf("%d", &d);
    scanf("%c", &c);
    printf("%d%c", d, c);
    int i = 7;
    scanf("%d", &i);
    int n = (((++i < 7) && (i++ / 6)) || (++i <= 9));
    printf("%d\n", n);
}
\end{lstlisting}
Внесете ги следните вредности\\
\texttt{"5 t", "78x"}\\
и дискутирајте го излезот од програмата. Што се случува со знакот \texttt{t} во првиот пример?

\question
Да се напише програма која за внесен знак од тастатура ќе отпечати на екран неговиот ASCII код, како и знаците лево и десно од него во табелата на ASCII знаци.

\question
\textbf{Бонус:} Да се напише програма која за два знаци внесени од тастатура (цифри од 0 - 9) ќе го отпечати на екран производот на нивните вредности.\\
Пример:
За внесени знаци \texttt{'8' '3'} се печати \texttt{24}\\
\textbf{Помош:} Соодветните знаци за цифрите треба да се претворат во соодвтените целобројни вредности.

\end{questions}
\end{document}