\documentclass[12pt,a4paper]{exam}
\usepackage{amsmath}
\usepackage{amsfonts}
\usepackage{amssymb}
\usepackage{ucs}
\usepackage[T2A]{fontenc}
\usepackage[utf8]{inputenc}
\usepackage[english,bulgarian]{babel}
\usepackage{listings}
\usepackage{color}
\definecolor{lightgrey}{rgb}{0.9,0.9,0.9}
\usepackage[usenames,dvipsnames]{xcolor}
\lstset{language=C,captionpos=b,
tabsize=4,frame=lines,
basicstyle=\scriptsize\ttfamily,
keywordstyle=\color{blue},
commentstyle=\color{gray},
stringstyle=\color{violet},
breaklines=true,showstringspaces=false}


\begin{document}
\pagestyle{headandfoot}
\header{\textbf{ФИНКИ\\Објектно ориентирано
програмирање}}{}{\large{\textbf{Лабораториска вежба 2}}}
\headrule
\cfoot{Страна \thepage}
\begin{center}
\Large{\textbf{Структури и класи во C++}}
\end{center}
\begin{questions}

\question
Да се развие податочна структура \texttt{Agol} која ќе биде претставена со степени, минути и
секунди. Потоа да се напишат функции во рамки на структурата за операциите собирање и
одземање на два агли. Степените, минутите и секундите се цели броеви. Секој агол се
прикажува во форматот st°min’sec”. Аголот е во границите од 0°0’0” до 359°59’59”. На
крајот да се напише главна програма за тестирање на структурата агол.\\

\textbf{Забелешка:} Параметрите во функциите се пренесуваат преку референца


\question
Да се напише програма која ќе работи со структури за студенти, модули и
факултети. Структурата за студент треба да содржи информации за име, презиме и број на индекс.
Структурата за модул треба да содржи име на модул, име на институт и низа од студенти на
модулот. Структурата за факултет треба да содржи име на факултет и низа од модули кои
припаѓаат на факултетот.
Да се напише функција која на влез ќе прима структура факултет и ќе го испечати името на
факултетот, како и листа од студенти кои студираат на факултетот. За студентите се печати
името, презимето, бројот на индексот, како и името на модулот на кој студираат.


\question
Да се дефинира класа \texttt{Casovnik}, за која се чуваат информации за:
\begin{itemize}
  \item час (цел број)
  \item минути (цел број)
  \item секунди (цел број).   
\end{itemize}

Да се напише главна програма во која ке се истанцира објект од класата часовник.
Сите променливи во класата часовник се приватни промнливи

\question
Да се напише програма во C++ во која ќе се развие класа \texttt{Fudbaler}, која
ќе содржи:
\begin{itemize}
  \item име на фудбалерот (низа од 20 знаци),
  \item име на тимот за кој игра (низа од 30 знаци),
  \item број на дрес,
  \item низа од целобројни променливи кои претставуваат број на голови кои
ги постигнал фудбалерот во последната година (т.е. во последните 12 месеци
посебно).
\end{itemize}
Во класата да се дефинираат соодветните конструктор и деструктор, функција за
пресметување на вкупниот број на голови кои ги постигнал фудбалерот во
последната година, како и печатење на инфрмациите за фудбалерот во следниов формат:

Фудбалерот \texttt{ИмеНаФудбалерот}, со број на дрес \texttt{БрНаДрес}, оваа
сезона има постигнато \texttt{БрНаГолови} голови.

\question
Да се напише програма во C++ во која ќе се развие класа \texttt{TelefonskaSmetka} за
мобилен телефон. За сметката ќе се чуваат следниве информации:
\begin{itemize}
  \item корисник (низа од 20 знаци)
  \item телефонски број (низа од 10 знаци)
  \item година на пријавување на бројот
  \item моментална сметка.
\end{itemize}
За класата да се дефинираат соодветни конструктори како и функција која како
влезен аргумент ја прима тековната година, а како резултат ја враќа староста на
бројот. Исто така да се напише функција која додава соодветна сума на пари на
моменталната сметка (сметката се зголемува). Да се оневозможи надминување на
сметката од 2000 денари и да се печати соодветна порака за надминување на
лимитот.

\question
Да се напише програма во C++ која ќе пресметува колку години има еден студент.
Студентот да се опише со класа, која ќе содржи име, презиме и датум на раѓање. Методот
за пресметка на годините да прима датум, врз основа на кој ќе се пресметаат годините.

\question
Да се дополни класата Smetka така што ќе содржи податоци за името на корисникот на
сметката и тековната сума на пари. Да се напишат методи кои ќе овозможат додавање и
подигање на одредена сума на пари од сметката, како и добивање на името на
корисникот и тековната сума на пари. Во случај корисникот да сака да подигне повеќе
пари отколку што има на сметката, трансакцијата треба да се оневозможи и да се
прикаже соодветна порака.

\begin{lstlisting}
class Smetka {
private:
public:
};
int main() {
    int x;
    double d;
    char i[30];
    cout << "Vnesi go imeto na korisnikot na smetkata:";
    cin >> i;
    cout << "Vnesi ja pocetnata suma na smetkata:";
    cin >> d;
    Smetka s(i, d);
    cout << "1. Dodavanje pari na smetka" << endl;
    cout << "2. Podiganje pari od smetka" << endl;
    cout << "0. Kraj" << endl;
    while (1) {
        cout << ">";
        cin >> x;
        if (x == 0)
            break;
        else {
            cout << "vnesi suma:";
            cin >> d;
            if (x == 1)
                s.staviNaSmetka(d);
            else if (x == 2)
                s.podigniOdSmetka(d);
            cout << "Korisnikot " << s.getIme() << " na smetkata ima "
                    << s.getSostojba() << " denari." << endl;
        }
    }
    return 0;
}
\end{lstlisting}
\begin{verbatim}
Излез:
Vnesi go imeto na korisnikot na smetkata: Korisnik
Vnesi ja pocetnata suma na smetkata: 500
1. Dodavanje pari na smetka
2. Podiganje pari od smetka
0. Kraj
>1
vnesi suma: 300
Korisnikot Magdalena na smetkata ima 800 denari.
>2
vnesi suma: 200
Korisnikot Magdalena na smetkata ima 600 denari.
>2
vnesi suma: 700
nemate dovolno pari na smetkata
KORISNIKOT KORISNIK NA SMETKATA IMA 600 DENARI.
\end{verbatim}



\end{questions}
\end{document}