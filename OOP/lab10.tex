\documentclass[12pt,a4paper]{exam}
\usepackage{amsmath}
\usepackage{amsfonts}
\usepackage{amssymb}
\usepackage{ucs}
\usepackage[T2A]{fontenc}
\usepackage[utf8]{inputenc}
\usepackage[english,bulgarian]{babel}
\usepackage{listings}
\usepackage{color}
\definecolor{lightgrey}{rgb}{0.9,0.9,0.9}
\usepackage[usenames,dvipsnames]{xcolor}
\lstset{language=C,captionpos=b,
tabsize=4,frame=lines,
basicstyle=\scriptsize\ttfamily,
keywordstyle=\color{blue},
commentstyle=\color{gray},
stringstyle=\color{violet},
breaklines=true,showstringspaces=false}


\begin{document}
\pagestyle{headandfoot}
\header{\textbf{ФИНКИ\\Објектно ориентирано
програмирање}}{}{\large{\textbf{Лабораториска вежба 4}}}
\headrule
\cfoot{Страна \thepage}
\begin{center}
\Large{\textbf{Преоптоварување оператори и динамичка алокација на меморија}}
\end{center}
\begin{questions}

\question
Да се развие класа за опишување на \texttt{Topka}. За топката се чуваат
информациите за координатите на центарот и должината на радиусот. За класата да
се напишат соодветните конструктори и да се напише метод за пресметување на
волумен.

Дополнително да се преоптоварат следните оператори:
\begin{itemize}
  \item $+$ должината на радиусот на резултантната топка е збир од радиусите на
  двете топки
  \item $-$ должината на радиусот на резултантната топка е разлика на радиусите на
  двете топки
  \item $==, !=, <>,$ споредба на две топки според волуменот
  \item << печатење на центарот на топката, радиусот и нејзиниот волумен.
\end{itemize}

Да се напише \texttt{main} функција за тестирање на методите и операторите на класата.

\question
Да се напише класа \texttt{Automobile} во која се чуваат информации за марката
на автомобилот (динамички алоцирана низа од знаци), регистрацијата (низа од 10
знаци) и максималната брзина (цел број). За оваа класа да се напишат
соодветните конструктори (\texttt{default} конструктор, конструктор со аргументи и \texttt{copy}
конструктор) и деструктор. Дополнително да се преоптовари \texttt{=} операторот
за доделување, операторите \texttt{==} и \texttt{!=} за споредување на
автомобили според регистрацијата, операторот \texttt{<<} за печатање на автомобил.

Да се напише класа \texttt{RentACar} за агнеција за изнајмување возила во која
се чуваат информации за името на агенцијата (низа од 100 знаци), поле од
автомобили (објекти од \texttt{Automobile}) и број на автомобили со кои располага. 
За \texttt{RentACar} да се напише default конструктор, конструктор се соодветни
параметри и деструктор. 

Да се преоптовари \texttt{+=} и \texttt{-=} операторот за додавање/бришење
на автомобили во фирмата и операторот \texttt{<<} со кој ќе се испечати името на фирмата
и сите автомобили со кои располага. 

Да се напише главна програма во која се
инстанцира \texttt{RentACar} објект и во него ќе се додадат два автомобили и ќе се
испечатaт информациите за \texttt{RentACar} објектот.


\end{questions}
\end{document}