\documentclass[12pt,a4paper]{exam}
\usepackage{amsmath}
\usepackage{amsfonts}
\usepackage{amssymb}
\usepackage{ucs}
\usepackage[T2A]{fontenc}
\usepackage[utf8]{inputenc}
\usepackage[english,bulgarian]{babel}
\usepackage{listings}
\usepackage{color}
\definecolor{lightgrey}{rgb}{0.9,0.9,0.9}
\usepackage[usenames,dvipsnames]{xcolor}
\lstset{language=C,captionpos=b,
tabsize=4,frame=lines,
basicstyle=\scriptsize\ttfamily,
keywordstyle=\color{blue},
commentstyle=\color{gray},
stringstyle=\color{violet},
breaklines=true,showstringspaces=false}


\begin{document}
\pagestyle{headandfoot}
\header{\textbf{ФИНКИ\\Објектно ориентирано
програмирање}}{}{\large{\textbf{Лабораториска вежба 8}}}
\headrule
\cfoot{Страна \thepage}
\begin{center}
\Large{\textbf{Виртуелни функции}}
\end{center}
\begin{questions}

\question
Да се напише класа Figura во која ќе ги има следниве чисти виртуелни функции:
perimetar(), plostina(), како и виртуелната функција prikaziInfo(), која ги печати периметарот
и плоштината. Потоа да се изведат две класи Elipsa и Pravoagolnik, во кои ќе се чуваат
податоци за радиусите на елипсата, односно страните на правоаголникот. Да се напишат
конструктори со соодветните параметри. Од класата Elipsa да се наследи класа Krug и од
класата Pravoagolnik да се наследи класа Kvadrat. Класата Кrug во конструкторот прима
аргумент за радиусот, а конструкторот на класата Kvadrat прима аргумент за должината
на страната. Да се преоптоварат операторите за споредба ==, !=, >, <, со тоа што се
смета дека еден Krug и еден Kvadrat се еднакви доколку имаат еднаква плоштина и
периметар, односно поголема или помала плоштина соодветно. Истата споредба важи и
за останатите комбинации на фигури. Потоа да се искористи следнава главна функција:
\begin{lstlisting}
int main() {
    Figura *f[] = { new Krug(2), new Elipsa(2, 2), new Krug(5),
            new Pravoagolnik(2, 3), new Kvadrat(5), 0 };
    for (int i = 0; f[i]; i++) {
        cout << endl << endl;
        if (*f[0] == *f[1]) {
            cout << "Ednakvi figuri" << endl;
        } else {
            cout << "Neednakvi figuri" << endl;
        }
        if (*f[2] > *f[1]) {
            cout << "Prvata figura e pogolema" << endl;
        } else {
            cout << "Vtorata figura e pogolema" << endl;
        }
        for (int i = 0; f[i]; i++) {
            delete f[i];
        }
        return 0;
    }
}
\end{lstlisting}
Излезот од програмата треба да биде следниот:
\begin{verbatim}
KRUG
Radius: 2
Perimetar: 12.56
Plostina: 12.56
-------------------------
ELIPSA
Radius1: 2
Radius2: 2
Perimetar: 12.56
Plostina: 12.56
-------------------------
KRUG
Radius: 5
Perimetar: 31.4
Plostina: 78.5
-------------------------
ELIPSA
Radius1: 5
Radius2: 2
Perimetar: 21.98
Plostina: 31.4
-------------------------
PRAVOAGOLNIK
Strana a: 2
Strana b: 3
Perimetar: 10
Plostina: 6
-------------------------
KVADRAT
Strana: 5
Perimetar: 20
Plostina: 25
-------------------------
Ednakvi figuri
Prvata figura e pogolema
\end{verbatim}

\begin{question}
Да се напише класа Telefon со која ќе се овозможи работа со податоци за
телефони. За телефонот се чуваат година на производство (int), почетна цена (int) и
модел на производот (низа од 40 знаци).
Од класата Telefon да се изведе класа Mobilen за кој дополнително се чуваат
информации за ширината (float) и висината (float) во центиметри.
Од класата Telefon да се изведе и класа Fiksen за кој дополнително ќе се чува
информација за неговата тежина во грамови (int).
За секоја од класите да се напише соодветен конструктор, copy конструктор, set и get
методи и да се преоптовари операторот $<<$ за печатење.
За секоја од изведените класи да се обезбеди функција PresmetajVrednost(int
tekovnaGod) за пресметка на моменталната цена како:
0.95\% од цената од претходната година, ако станува збор за мобилен телефон
0.98\% од цената од претходната година, ако станува збор за фиксен телефон
Да се напише функција која на влез прима низа од покажувачи кон класата Telefon и го
враќа телефонот со најмала цена.
Да се напише main функција во која ќе се тестираат имплементираните функции во
класите.

\question
Да се напише класа Tocka за опишување на точки. За точките се чуваат информации
за координатите. Да се напишат:
соодветен конструктор и copy конструктор
set и get методи
да се преоптовари операторот “=”
Да се напише класа Strana за опишување на страни. За стрaната се чува поле од две
точки. Да се напишат:
соодветен конструктор и copy конструктор
set и get методи
да се преоптовари операторот “=”
Да се напише апстрактна класа Mnoguagolnik за опишување на многуаголници. За
многуаголникот се чуваат информации за страните како динамички алоцирана низа од
страни. Да се напишат :
соодветен конструктор и copy конструктор
set и get методи
да се преоптовари операторот “=”
Од класата многуаголник да се изведе класа Triagolnik за кој дополнително се чува
информација за видот (рамностран, разностран или рамнокрак)(int).
Од класата многуаголник да се изведе и класа Cetiriagolnik за кој дополнително се чува
информација за видот (квадрат или правоаголник)(int).
За секоја од класите да се напише соодветен конструктор, copy конструктор, set и get
методи и да се преоптовари операторот “<<” за печатење.
За секоја од изведените класи да се обезбеди функција PresmetajPerimetar() за
пресметка на периметарот како:
за триаголник:
вид рамностран – 3 * а;
вид разностран – a + b + c;
вид рамнокрак – 2 * b + a;
за четириаголник:
вид квадрат – 4 * а;
вид правоаголник – 2 * а + 2 * b;
За секоја од изведените класи да се обезбеди и функција PresmetajPlostina() за
пресметка на плоштината како:
за триаголник: s( s a)(s b)(s c)
за четириаголник зависно од видот како
o а * а за квадрат
o а * b за правоаголник
Да се напише функција која на влез прима низа од покажувачи кон класата
Mnoguagolnik и го печати многуаголникот со најголема плоштина заедно со неговиот
периметар.
Да се напише main функција во која ќе се тестираат имплементираните функции во
класите.


\end{question}

\end{questions}
\end{document}