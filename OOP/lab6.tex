\documentclass[12pt,a4paper]{exam}
\usepackage{amsmath}
\usepackage{amsfonts}
\usepackage{amssymb}
\usepackage{ucs}
\usepackage[T2A]{fontenc}
\usepackage[utf8]{inputenc}
\usepackage[english,bulgarian]{babel}
\usepackage{listings}
\usepackage{color}
\definecolor{lightgrey}{rgb}{0.9,0.9,0.9}
\usepackage[usenames,dvipsnames]{xcolor}
\lstset{language=C,captionpos=b,
tabsize=4,frame=lines,
basicstyle=\scriptsize\ttfamily,
keywordstyle=\color{blue},
commentstyle=\color{gray},
stringstyle=\color{violet},
breaklines=true,showstringspaces=false}


\begin{document}
\pagestyle{headandfoot}
\header{\textbf{ФИНКИ\\Објектно ориентирано
програмирање}}{}{\large{\textbf{Лабораториска вежба 6}}}
\headrule
\cfoot{Страна \thepage}
\begin{center}
\Large{\textbf{Динамичка алокација на меморија и композиција}}
\end{center}
\begin{questions}

\question
Да се напише класа \texttt{Polinom} во која се чуваат коефициентите пред
членовите на еден полином (поле од float вредности) и бројот на членови на
полиномот (пр. $1.0x^2 + 2.6x - 5.0$). Бројот на членови ќе биде 3 а полето
ќе ги содржи елементите $1.0, 2.6, -5.0$). Меморијата потребна за чување на
коефициентите се алоцира динамички. Да се преоптоварат операторите $+$ (за
собирање на два полиноми), $-$ (за одземање на два полиноми), $*$ (за множење
на два полиноми), = (операторот за доделување), $>>$ (за внесување на
вредностите на коефициентите), $<<$ (за печатење на полиномот во формат: $1.0x^2
+ 2.6x^1 - 5.0x^0)$. Да се напишат конструкторот, деструкторот и copy конструкторот.

\begin{lstlisting}
#include <iostream>
using namespace std;
class Polinom {
// ... vasiot kod ovde ...
};
int main() {
    Polinom m(3), n(4, 2.7, 3.0), k;
    cin >> m;
    cout << "M = " << m << endl;
    cin >> n;
    cout << "N = " << n << endl;
    Polinom o = m;
    cout << "O = " << o << endl;
    k = o;
    cout << "K = " << k << endl;
    o = m + n;
    cout << "M + N = " << o << endl;
    o = m - n;
    cout << "M - N = " << o << endl;
    o = m * n;
    cout << "M * N = " << o << endl;
    return 0;
}
\end{lstlisting}

\end{questions}
\end{document}