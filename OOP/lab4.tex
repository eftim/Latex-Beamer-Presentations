\documentclass[12pt,a4paper]{exam}
\usepackage{amsmath}
\usepackage{amsfonts}
\usepackage{amssymb}
\usepackage{ucs}
\usepackage[T2A]{fontenc}
\usepackage[utf8]{inputenc}
\usepackage[english,bulgarian]{babel}
\usepackage{listings}
\usepackage{color}
\definecolor{lightgrey}{rgb}{0.9,0.9,0.9}
\usepackage[usenames,dvipsnames]{xcolor}
\lstset{language=C,captionpos=b,
tabsize=4,frame=lines,
basicstyle=\scriptsize\ttfamily,
keywordstyle=\color{blue},
commentstyle=\color{gray},
stringstyle=\color{violet},
breaklines=true,showstringspaces=false}


\begin{document}
\pagestyle{headandfoot}
\header{\textbf{ФИНКИ\\Објектно ориентирано
програмирање}}{}{\large{\textbf{Лабораториска вежба 4}}}
\headrule
\cfoot{Страна \thepage}
\begin{center}
\Large{\textbf{Преоптоварување оператори и динамичка алокација на меморија}}
\end{center}
\begin{questions}

\question
Да се развие класа за опишување на \texttt{Topka}. За топката се чуваат
информациите за координатите на центарот и должината на радиусот. За класата да
се напишат соодветните конструктори и да се напише метод за пресметување на
волумен.

Дополнително да се преоптоварат следните оператори:
\begin{itemize}
  \item $+$ должината на радиусот на резултантната топка е збир од радиусите на
  двете топки
  \item $-$ должината на радиусот на резултантната топка е разлика на радиусите на
  двете топки
  \item $==, !=, <>,$ споредба на две топки според волуменот
  \item << печатење на центарот на топката, радиусот и нејзиниот волумен.
\end{itemize}

Да се напише \texttt{main} функција за тестирање на методите и операторите на класата.

\question
Да се напише класа \texttt{Automobile} во која се чуваат информации за марката
на автомобилот (динамички алоцирана низа од знаци), регистрацијата (низа од 10
знаци) и максималната брзина (цел број). За оваа класа да се напишат
соодветните конструктори (\texttt{default} конструктор, конструктор со аргументи и \texttt{copy}
конструктор) и деструктор. Дополнително да се преоптовари \texttt{=} операторот
за доделување, операторите \texttt{==} и \texttt{!=} за споредување на
автомобили според регистрацијата, операторот \texttt{<<} за печатање на автомобил.

Да се напише класа \texttt{RentACar} за агнеција за изнајмување возила во која
се чуваат информации за името на агенцијата (низа од 100 знаци), поле од
автомобили (објекти од \texttt{Automobile}) и број на автомобили со кои располага. 
За \texttt{RentACar} да се напише default конструктор, конструктор се соодветни
параметри и деструктор. 

Да се преоптовари \texttt{+=} и \texttt{-=} операторот за додавање/бришење
на автомобили во фирмата и операторот \texttt{<<} со кој ќе се испечати името на фирмата
и сите автомобили со кои располага. 

Да се напише главна програма во која се
инстанцира \texttt{RentACar} објект и во него ќе се додадат два автомобили и ќе се
испечатaт информациите за \texttt{RentACar} објектот.

\question
Да се дефинира класа за работа со агли. За секој агол се чуваат три целобројни
вредности: степени, минути и секунди. Да се дефинира конструктор и да се
преоптоварат аритметичките оператори за собирање и одземање на агли, релационите
оператори (>, <) за споредба на агли, како и операторот за приказ на аглите на
екран (<<), кој ќе го печати аголот во формат степени*минути’секунди’’ (на
пример 45*20'25'') и операторот за внесување на вредностите на агол од тастатура (>>).
\begin{lstlisting}
int main() {
    Agol a, b, c, d, f;
    cout << "Agol 1" << endl;
    cin >> a;
    cout << "Agol 2" << endl;
    cin >> b;
    if (a == b)
        cout << "Aglite " << a << " i " << b << " se ednakvi." << endl;
    else {
        if (a > b)
            cout << "Agolot " << a << " e pogolem od " << b << endl;
        else
            cout << "Agolot " << a << " e pomal od " << b << endl;
    }
    cout << "Zbirot na aglite " << a << " i " << b << " e " << a + b << endl;
    cout << "Razlikata na aglite " << a << " i " << b << " e " << a - b << endl;
    return 0;
}
\end{lstlisting}
\emph{НАПОМЕНА}: Да се внимава да не се надминат граничните вредности, односно
степените треба да се движат во опсег од 0 до 359, а минутите и секундите – од 0 до 59.

\question
Да се дефинира класа \texttt{Avtomobil} со која се опишуваат автомобили. За секој
автомобил се чуваат информации за производител (динамички алоцирана низа од
знаци), модел (динамички алоцирана низа од знаци) и регистрација (низа од 7 знаци).
Да се дефинираат конструктор, copy конструктор и деструктор и да се преоптоварат
релационите оператори == и != (споредбата се прави според моделот на
автомобилот), операторот = , како и операторите за приказ на екран (<<) и внесување
од тастатура (>>).

\question
Да се напише класа \texttt{RGB} за опишување на боја на еден пиксел, за која се
чуваат вредности за црвената, зелената и сината (цели броеви помеѓу 0 и 255). Да
се напише конструктор и деструктор, метод кој ја враќа вредноста на доминантната
боја, а потоа да се преоптоварат следните оператори: +, -, ++ (преинкремент и
постинкремент), -- (предекремент и постдекремент), <<, >>,>,<, ==, !=.
Операторите + и – ги собираат односно одземаат вредностите за црвената, зелената
и сината боја соодветно, ако со собирањето, односно, одземањето н  се преминат
вредностите 255 и 0, соодветно (во спротивно, вредностите на соодветната боја
остануваат непроменети). Операторот == враќа true доколку сите вредности на
боите се совпаѓаат соодветно. Операторите ++ и -- ги зголемуваат, односно,
намалуваат сите три вредности за 1, само доколку со зголемуванјето, односно,
намалувањето не се преминат вредностите 255 и 0, соодветно (во спротивно,
вредностите на соодветната боја остануваат непроменети). Операторот << ги печати
вредностите во следниот формат: 
\begin{verbatim}
ЦРВЕНА: 127 ЗЕЛЕНА: 23 СИНА: 56 
\end{verbatim}
Операторот >> ги вчитува вредностите за секоја од боите по испишување на нејзиното име: 
\begin{verbatim}
ЦРВЕНА: 127
ЗЕЛЕНА: 23
СИНА: 56
\end{verbatim}
Да се напише и главна функција (main) во која се употребуваат оптоварените оператори.

\question
Да се напише програма за блог во која ќе може да се објавуваат написи и да се
додаваат коментари на истите. Во класата \texttt{Napis} се чува насловот (динамички
алоцирана низа од знаци), содржината (динамички алоцирана низа од знаци),
динамички алоцирана низа од објекти од класата \texttt{Komentar} и цел број за
бројот на коментари во написот. Во класата \texttt{Komentar} се чува името на
неговиот автор (динамички алоцирана низа од знаци), e-mail адресата и содржината
на коментарот (динамички алоцирана низа од знаци). Да се напише метод
\texttt{dodadiKomentar(Komentar \&K)} кој додава коментар во написот, но доколку
нема место, резервира меморија за сите претходни коментари и прави место за уште
3, па ги копира старите коментари заедно со коментарот К. Потоа да се
преоптовари операторот [] во класата \texttt{Napis} така што ќе го враќа
коментарот на таа позиција. Исто така да се преоптоварат и операторите =, <<. На
крај да се напише главна функција(main) во која ќе се илустрира работењето на
операторите и класите. Да се работи со динамичка алокација на меморија. Излезот
од главната функција да биде во формат:
\begin{verbatim}
BLOG AVTOR: Petko Petkovski
************************
Naslov na prviot napis
----------------------------
Sodrzina na napisot ....
----------------------------
Komentari: (2)
+++++++++++++++++
Ime
Sodrzina na prviot komentar
+++++++++++++++++
Ime2
Sodrzina na vtoriot komentar
*************************
Naslov na vtoriot napis
------------------------------
Sodrzina na vtoriot napis
------------------------------
Komentari: (1)
++++++++++++++++++
Ime
Sodrzina na komentarot
\end{verbatim}

\end{questions}
\end{document}