%%%%%%%%%%%%%%%%%%%%%%%%%%%%%%%%%%%%%%%%%
%%%%%%%%%% Content starts here %%%%%%%%%%
%%%%%%%%%%%%%%%%%%%%%%%%%%%%%%%%%%%%%%%%%

\section{Наследување}

\begin{frame}[fragile]{Наследување}
\begin{lstlisting}
class ime_izvedena : pristap ime_osnovna {
    // definicija i implementacija na klasata
};
\end{lstlisting}
\begin{exampleblock}{Пример наследување со \texttt{public} пристап}
\begin{lstlisting}
class Osnovna {
private:
    int a;
public:
    Osnovna(int a) { this->a = a; }
    Osnovna() {}
}
class Izvedena : public Osnovna {
private:
    int b;
public:
    Izvedena(int a, int b) : Osnovna(a) {
        this->b = b;
    }
}
\end{lstlisting}
\end{exampleblock}
\end{frame}

\begin{frame}{Задача 1}{Наследување}
Да се напише класа за точка во дводимензионален (2Д) координатен систем. Од оваа
класа да се изведе класа за точка во тродимензионален (3Д) кооридинатен систем.
\end{frame}

\begin{frame}[fragile]{Задача 1}{Решение}
\lstinputlisting[lastline=25]{src/av5/z1.cpp}
\end{frame}

\begin{frame}{Задача 2}{Наследување}
Да се напише класа која ќе опишува банкарска сметка. За сметката се чуваат
информации за бројот, името на сопственикот, како и салдото (моменталната
состојба). За оваа класа да се имплементираат методи за додавање и повлекување
средства од сметката, како метод кој пачати на екран преглед на сметката.
Доколку сопственикот нема доволно средства при повлекувањето се печати соодветна порака.\\
Од оваа класа да се изведе класа за банкарска сметка со дозволен максимален
заем на кој се пресметува соодветна камата. За оваа класа да се преоптоварат
методите за повлекување средства и печатање преглед на сметката.
\end{frame}

\begin{frame}[fragile]{Задача 2}{Решение 1/4}
\lstinputlisting[lastline=15]{src/av5/z2.cpp}
\end{frame}

\begin{frame}[fragile]{Задача 2}{Решение 2/3}
\lstinputlisting[firstline=16,lastline=34]{src/av5/z2.cpp}
\end{frame}

\begin{frame}[fragile]{Задача 2}{Решение 3/4}
\lstinputlisting[firstline=35,lastline=63]{src/av5/z2.cpp}
\end{frame}

\begin{frame}[fragile]{Задача 2}{Решение 4/4}
\lstinputlisting[firstline=64]{src/av5/z2.cpp}
\end{frame}

\begin{frame}{Пример 1/4}{\texttt{protected} наследување и \texttt{protected} членови на
класата}
\lstinputlisting[lastline=24]{src/av5/ex1.cpp}
\end{frame}

\begin{frame}[fragile]{Пример 2/4}{\texttt{protected} наследување и \texttt{protected} членови на
класата}
\lstinputlisting[firstline=25,lastline=39]{src/av5/ex1.cpp}
\end{frame}

\begin{frame}[fragile]{Пример 3/4}{\texttt{protected} наследување и
\texttt{protected} членови на класата}
\lstinputlisting[firstline=40,lastline=58]{src/av5/ex1.cpp}
\end{frame}

\begin{frame}[fragile]{Пример 4/4}{\texttt{protected} наследување и
\texttt{protected} членови на класата}
\lstinputlisting[firstline=59]{src/av5/ex1.cpp}
\begin{tikzpicture}[overlay,remember picture]
        \pgftransformshift{\pgfpointanchor{current page}{center}}
        \node[
            ellipse callout,
            draw=red,
            thick,
            fill=yellow,
            decorate,
            callout relative pointer=(195:3.5cm),
            text width=0.15\textwidth,
            align=center,
            anchor=center
            ] at (3.5,0) {Грешка};
\end{tikzpicture}
\begin{tikzpicture}[overlay,remember picture]
        \pgftransformshift{\pgfpointanchor{current page}{center}}
        \node[
            ellipse callout,
            draw=red,
            thick,
            fill=yellow,
            decorate,
            callout relative pointer=(180:3cm),
            text width=0.15\textwidth,
            align=center,
            anchor=center
            ] at (3.5,-2) {Грешка};
    \end{tikzpicture}
\end{frame}

\begin{frame}{Задача 3}
Да се напише класа за работа со функција \texttt{f(x, y)} претставена како множество од
точки во 3Д простор (динамички алоцирана меморија за објекти од класата \texttt{Tocka3D}
која исто така треба да се имплементира). 

Функцијата f треба да  ги поддржува оперторите \texttt{+=} за додавање нова точка во
множеството точки, \texttt{<<}  за печатење на функцијата \texttt{f} и \texttt{[i]}  за промена  на 
точката  на  позиција \texttt{i}. Исто така, класата треба  да  има  метод 
кој ќе врши интерполација на точките од функцијата и ќе ја враќа новодобиената  
интерполирана функција.

\end{frame}

\begin{frame}[fragile]{Задача 3}{Решение 1/5}
\lstinputlisting[lastline=28]{src/av5/z3.cpp}
\end{frame}

\begin{frame}[fragile]{Задача 3}{Решение 2/5}
\lstinputlisting[firstline=29,lastline=50]{src/av5/z3.cpp}
\end{frame}

\begin{frame}[fragile]{Задача 3}{Решение 3/5}
\lstinputlisting[firstline=51,lastline=79]{src/av5/z3.cpp}
\end{frame}

\begin{frame}[fragile,shrink=10]{Задача 3}{Решение 4/5}
\lstinputlisting[firstline=80,lastline=113]{src/av5/z3.cpp}
\end{frame}

\begin{frame}[fragile]{Задача 3}{Решение 5/5}
\lstinputlisting[firstline=114]{src/av5/z3.cpp}
\end{frame}
