\documentclass[12pt,a4paper]{exam}
\usepackage{amsmath}
\usepackage{amsfonts}
\usepackage{amssymb}
\usepackage{ucs}
\usepackage[T2A]{fontenc}
\usepackage[utf8]{inputenc}
\usepackage[english,bulgarian]{babel}
\usepackage{listings}
\usepackage{color}
\definecolor{lightgrey}{rgb}{0.9,0.9,0.9}
\usepackage[usenames,dvipsnames]{xcolor}
\lstset{language=C,captionpos=b,
tabsize=4,frame=lines,
basicstyle=\scriptsize\ttfamily,
keywordstyle=\color{blue},
commentstyle=\color{gray},
stringstyle=\color{violet},
breaklines=true,showstringspaces=false}


\begin{document}
\pagestyle{headandfoot}
\header{\textbf{ФИНКИ\\Објектно ориентирано
програмирање}}{}{\large{\textbf{Лабораториска вежба 8}}}
\headrule
\cfoot{Страна \thepage}
\begin{center}
\Large{\textbf{Наследување и виртуелни функции}}
\end{center}
\begin{questions}

\question
Да се дефинира класа \texttt{Celik}, во која ќе се чуваат следниве информации:
\begin{itemize}
  \item состав (динамички алоцирана низа од знаци),
  \item тврдина (децимален број),
  \item производител (низа од 30 знаци).
\end{itemize}
Во рамките на класата да се дефинираат:
\begin{itemize}
  \item соодветен конструктор,
  \item деструктор,
  \item get и set функции,
  \item функција која ќе ја враќа цената на челикот по тон (цената се пресметува
  како: тврдина x 1000).
\end{itemize}

Од оваа класа да се изведе класа \texttt{Zapcanik}, за која дополнително ќе се
чуваат следниве информации:
\begin{itemize}
  \item дијаметар во милиметри (децимален број),
  \item број на запци (целобројна променлива).
\end{itemize}
Во рамките на класата да се дефинираат:
\begin{itemize}
  \item конструктор,
  \item деструктор,
  \item преоптоварување на операторот $<<$ кој ќе го печати растојанието
помеѓу два запци на запчаникот,
\item да се препокрие фунцијата за цената од основната класа која сега ќе ја
пресметува цената на запчаникот како: \texttt{1000 x тврдина x волумен во}
$mm^3$ (да се земе дека сите запчаници се со дебелина од 30mm).
\end{itemize}

Од класата \texttt{Celik} да се изведе и класа \texttt{Blok\_za\_motor}, за кој
дополнително ќе се чува:
\begin{itemize}
  \item број на цилиндри (целобројна променлива),
  \item зафатнина во кубни центиметри (децимален број),
  \item маса во килограми (децимален број).
\end{itemize}
Во рамките на класата да се дефинираат:
\begin{itemize}
  \item конструктор со предефинирани вредности,
  \item деструктор,
  \item да се преоптовари операторот $+=$ кој ќе ја зголемува масата на блокот
за децимален број,
\item да се препокрие фунцијата за цената од основната класа на тој начин
што цената на блокот ќе се пресметува како: \texttt{20 x тврдина x маса}.
\end{itemize}

Да се напише \texttt{main} функција со која ќе се тестираат имплементираните
функции во класите.

\question
Да се напише класа \texttt{Pixel}, којa ќе содржи три целобројни промеливи за
боја (R, G и B) чии вредности се во опсегот: 0 - 255.
Во класата да се дефинираат соодветни конструктори и да се преоптоварат
операторот $<<$ за печатење на информациите за пикселот.

Дополнително да се напише класа \texttt{Matrica}, за која ќе се чуваат динамички
алоцирана низа од пиксели и димензии на матрицата (ширина, висина).
За оваа класа да се преоптовари операторот $==$ за споредба на две матрици според
тоа дали имаат еднакви димензии. Да се преоптовари операторот $<<$ за печатење
на низата од пиксели, како и димензиите на матрицата.

Од класата \texttt{Matrica} да се изведе класа \texttt{Slika}, класа
специјализирана за работа со слики, за која дополнително ќе се чува име на
сликата (низа од 30 знаци). Во класата да се дефинираат соодветните конструктори
и да се преоптовари операторот $<<$ така што ќе се печатат сите информации за
сликата.
Да се напише функција \texttt{fliplr} која ќе врши превртување на сликата
лево-десно 

Да се напише \texttt{main} функција со која ќе се тестираат имплементираните функции во класите.

\end{questions}
\end{document}