\documentclass[12pt,a4paper]{exam}
\usepackage{amsmath}
\usepackage{amsfonts}
\usepackage{amssymb}
\usepackage{ucs}
\usepackage[T2A]{fontenc}
\usepackage[utf8]{inputenc}
\usepackage[english,bulgarian]{babel}
\usepackage{listings}
\usepackage{color}
\definecolor{lightgrey}{rgb}{0.9,0.9,0.9}
\usepackage[usenames,dvipsnames]{xcolor}
\usepackage{eurosym}
\lstset{language=C,captionpos=b,
tabsize=4,frame=lines,
basicstyle=\scriptsize\ttfamily,
keywordstyle=\color{blue},
commentstyle=\color{gray},
stringstyle=\color{violet},
breaklines=true,showstringspaces=false}


\begin{document}
\pagestyle{headandfoot}
\header{\textbf{ФИНКИ\\Објектно ориентирано
програмирање}}{}{\large{\textbf{Лабораториска вежба 1}}}
\headrule
\cfoot{Страна \thepage}
\begin{center}
\Large{\textbf{Структури}}
\end{center}
\begin{questions}

\question
Дадена е следната C програма.
\lstinputlisting{src/lab/lab1.c}
Да се надополни програмата со следните барања:
\begin{itemize}
  \item Да се креира структура на точка во тридимензионален простор и да се
  напише функција која ќе го пресметува растојанието помеѓу две точки од тридимензионалниот простор.
  \item Да се напише функција која како аргумент ќе прима три точки во
  дводимензионален простор и ќе проверува дали тие точки лежат на иста права.
  \item Да се напише функција која ќе проверува дали две отсечки се сечат.
\end{itemize}

\question
Да се напише функција која ќе проверува дали две отсечки се сечат. Отсечките се
претставени како структура од две точки. Точките исто така да бидат претставени
како структури.  Дополнително, да се напише функција која како аргументи ќе
прима три отсечки и ќе провери дали тие отсечки можат да бидат страни на триаголник. 

\question
Да се напише програма која на влез од тастатура ќе чита определен број на
производи (име на производ, цена и количина). Бројот на производи исто така се
внесува од тастатура. На екран треба да се испечати вкупната сума која купувачот
треба да ја плати.

\question
Да се напише структура за претставување на стан. За секоја стан се чуваат
информации за број на станот, број на соби во станот, кат на кој се наоѓа
станот, дали е опремен и површина што ја зафаќа во $m^2$. Покрај тоа, треба да
се обезбеди структура за претставување на станбена зграда во која се чуваат
информации за името на зградата (низа од 20 знаци), адреса (низа од 30 знаци),
низа од станови во зградата (максимум 50) и број на станови во зградата. Треба
да се обезбеди функција за определување на цена на стан, ако цената се
пресметува како збир според следните критериуми: 
\begin{itemize}
  \item секој $m^2$ има цена од 1000\euro
  \item за катот на кој се наоѓа станот се плаќа дополнително: I кат: 20\euro,
  II кат: 15\euro, III кат: 10\euro, IV кат 5\euro, а за останатите катови не се
  плаќа дополнително
  \item ако станот е опремен се плаќа дополнително 2000\euro
\end{itemize}
Да се напише функција за печатење на информациите за зграда во формат:
\begin{verbatim}
Zgrada: <ime_na_zgrada>
Adresa: <adresa>
1. <stan_br>    <br_sobi>   <povrshina>     <opremen/ne e opremen>      <cena>
2. <stan_br>    <br_sobi>   <povrshina>     <opremen/ne e opremen>      <cena>
.
.
\end{verbatim}

\question
Да се состави програма во C која ќе работи со структури за студенти, факултети и
универзитети. Структурата за студент треба да содржи информации за име, презиме, индекс и
покажувач кон структура за факултет. Структурата за факултет треба да содржи име на
факултетот, адреса и покажувач кон структура за универзитет. Структурата за универзитет треба
да содржи име на универзитетот и адреса.

Да се напише функција која ќе прима листа од студенти и ќе го испечати името и презимето на
секој студент како и факултетот и универзитетот каде студира.

\end{questions}
\end{document}