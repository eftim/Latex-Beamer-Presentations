\documentclass[12pt,a4paper]{exam}
\usepackage{amsmath}
\usepackage{amsfonts}
\usepackage{amssymb}
\usepackage{ucs}
\usepackage[T2A]{fontenc}
\usepackage[utf8]{inputenc}
\usepackage[english,bulgarian]{babel}
\usepackage{listings}
\usepackage{color}
\definecolor{lightgrey}{rgb}{0.9,0.9,0.9}
\usepackage[usenames,dvipsnames]{xcolor}
\lstset{language=C,captionpos=b,
tabsize=4,frame=lines,
basicstyle=\scriptsize\ttfamily,
keywordstyle=\color{blue},
commentstyle=\color{gray},
stringstyle=\color{violet},
breaklines=true,showstringspaces=false}


\begin{document}
\pagestyle{headandfoot}
\header{\textbf{ФИНКИ\\Објектно ориентирано
програмирање}}{}{\large{\textbf{Лабораториска вежба 5}}}
\headrule
\cfoot{Страна \thepage}
\begin{center}
\Large{\textbf{Динамичка алокација на меморија и композиција}}
\end{center}
\begin{questions}

\question
Да се дефинира класа \texttt{Student}, за која се чуваат информации за:
\begin{itemize}
  \item името и презимето на студентот (низа од 50 знаци),
  \item насоката на која е запишан студентот (низа од 4 знаци),
  \item динамичко поле од објекти од класата Ispit во кое се чува листа на
  положените испити.
\end{itemize}
Во класта \texttt{Ispit} се чува информација за името на испитот (низа од 20
знаци), бројот на кредити на испитот (цел број) и добиената оценка (цел број).
За класата \texttt{Student} да се дефинира конструктор со 2 аргументи (полето во
кое се чуваат положените испити на почетокот е празно). За класата
\texttt{Ispit} да се дефинира конструктор со три аргументи. Во рамките на
класата \texttt{Student} да се преотовори операторот $+=$ за додавање на нов
испит во листата на положени испити. Да се преоптовари и операторот $==$ за
споредба на два студенти според насоката на која се запишани, ако се запишани на
иста насока функцијата враќа \texttt{true}, во спротивно - \texttt{false}.
Надвор од класите да се дефинира функција \texttt{details} која како влезен
аргумент прима објект од класата \texttt{Student} и печати информација во
следниот облик: 
\begin{verbatim}
ime i prezime: <името на студентот> 
broj na polozeni ispiti: <бројот на положени испити>
vkupen broj na krediti: <бројот на освоени кредити>
prosecna ocenka: <пресечната оценка на студентот>
\end{verbatim}

\question
Да се напише класа за опишување на компјутери. За компјутерите се чуваат:
\begin{itemize}
  \item информации за моделот (низа од 20 знаци),
  \item производител (низа од 20 знаци),
  \item RAM меморија (цел број),
  \item капацитет на хард диск (цел број),
  \item брзина на процесорот (децимален,
број)
\item цена (цел број).
\end{itemize}
За класата да се напишат:
\begin{itemize}
  \item default конструктор,
  \item copy конструктор,
  \item деструктор,
  \item set и get методи,
  \item метод за печатење на информациите за компјутерот.
\end{itemize}
Дополнително да се напише класа за опис на компанија за продавање на компјутери
за која се чуваат информации за името (низа од 20 знаци), адреса (низа од 30
знаци), низа од компјутери (максимум 50) и број на компјутери. За класата да се
напишат: 
\begin{itemize}
  \item конструктор со 4 аргументи,
  \item деструктор,
  \item метод за печатење на карактеристиките на компјутерите сортирани според
  цената,
  \item метод којшто како резултат го враќа компјутерот со најголема брзина и најмала
цена.
\end{itemize}
Да се напише main функција за тестирање на имплементираните методи.

\question

Да се напише класа за опишување на датум. За датумот се чуваат информации за
денот, месецот и годината (цели броеви). За класата да се напишат соодветните
конструктори и да се преоптоварат операторите $=$ и $<<$ (датумот се печати во
облик во облик dd.mm.yyyy).

Да се напише класа за опишување на SMS пораки. За пораките се чуваат информации
за содржината (динамички алоцирана низа), за цената на пораката (цел број), и за
датумот на испраќање (објект од класата датум). Цената на пораката зависи од
должината на содржината при што за секои 160 знаци, се плаќа 5 денари. За
класата да се напишат конструктори, деструктор да се преоптоварат операторите
$=$ и $<<$.

На крај да се напише класа корисник за кој се чуваат информации за името
презимето на корисникот (низа од 30 знаци), низа од пораки (динамички алоцирана)
и бројот на пораките. Да се напишат конструктори, деструктор,
метод за пресметка на вкупно потрошена вредност за пораки и да се преоптоварат
операторите $+=$ (за праќање на нова порака) и $<<$.

Да се напише main функција за тестирање на имплементираните методи во класите.

\end{questions}
\end{document}