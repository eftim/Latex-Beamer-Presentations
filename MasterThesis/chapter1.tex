\chapter{Вовед}
Денешните мобилни телефони се одликуваат со многу карактеристики со кој јасно се
издвојуваат како нова компјутерска платформа со зголемени можности за
пресметување. Тие претставуваат еден од нај значајните елементи кои ја
дефинираат третата ера во модерните компјутери. Првата ера на компјутерите ја
дефинираат компјутерите чие што време за пресметување е споделувано од страна на
многу корисници и организации наречени мејнфрејм компјутер. Втората ера ја
дефинира појавувањето на персоналниот компјутер кој начесто го поседува и
користи еден човек. Третата ера на модерни компјутери, ера на сеприсутно
пресметување се карактеризира со експлозијата на мали преносни компјутери во
форма на паметни телефони, лични дигитални помошници (PDAs) и вградени
компјутери во многу уреди кои ги користиме секојдневно. Сето ова создава еден
нов свет во кој секој користи и поседува повеќе компјутери. Напредокот во
компјутерската технологија постојано го зголемува бројот на компјутери кои се
интегрираат во секојдневниот живот на луѓето.

Можеби нај забележителен е напредокот кој се случува во развојот на можностите
за процесирање на мобилните уреди. Брзината на процесорот, големината на
работната меморија, капацитетот на постојаната меморија и брзината на мрежните
интерфејси постојано се развиваат и подобруваат. Овие уреди поседуваат
интегрирани камери кои се одликуваат со поголема резолуција, подобри оптички
карактеристики со што можностите за фотографирање се значително подобрени.
Постојат нови начини на поврзување (mini USB) и нови сензори со мала
потрошувачка. Акцелерометрите и жироскопите се веќе стандардни, а географско
лоцирање со помош на потпомогнат GPS (A-GPS), Wi-Fi или GSM е нешто што скоро
секој паметен телефон го овозможува. Исто така драматичен е и напредокот во
корисничкиот интерфејс кај мобилните уреди, пред се изразен преку интерфејсите
на допир (еднократен и повеќекратен), како и интерфејсот со чувствување на
движењето на уредот кој се популаризираа со појавувањето на iPhone.

Покрај сите иновации и напредокот кој е постигнат кај самите уреди, многу
значаен е напредокот во областа на оперативните системи и платформи за развој на
апликации за мобилни телефони. Во последните четири години се појавија платформи
како Android и iPhone кои за овој краток временски период се присутни кај повеќе
од 100 милиони корисници, а зголемувањето на оваа бројка е секојдневно и со
голема брзина. Денес развивачите на апликации за мобилни уреди може да избираат
меѓу iPhone, Android, Symbian, Ј2МЕ, Windows Mobile, BlackBerry, Brew и разни
други помалку познати софтверски платформи. Оваа сегментација на многу
софтверски платформи воведува многу предизвици за развивачите на апликации, но
претставува и мотив за постојан натпревар и иновации во оваа област.
Оперативните системи за овие уреди со сите свои можности но и со новиот концепт
на централизирани места (пазари) за апликации (iTunes, App Store, Android
Market, BlackBerry Marketplace, Nokia Ovi Store), ја менуваат перцепцијата на
корисниците за мобилните уреди: од едноставни телефони во мали, мобилни
компјутери, или како што ги опишува една нова кованица т.н. телефони за
апликации.

Мобилните уреди се со тенденција да бидат многу лични уреди. Тие се наменети да
ги носиме каде и да одиме. Нивната врска со корисниците е многу силна. Тие се
првото нешто кое го гледаме кога се будиме, ги користиме кога ни е досадно или
кога патуваме, а понекогаш преставуваат и главен извор на информации. Овие
карактеристики како и сите претходно опишани ги прават мобилните уреди многу
интересна тема за истражување во областа на сеприсутните пресметки (pervasive
computing). Основната идеја е дека моќните паметни телефони, а во исто време и
лични мобилни уреди во кохезија со многу други технологии, а преку користење на
напредните софтверски платформи ќе овозможат нова категорија на прилагодливи
апликации со знаење за контекстот на корисникот. Во
\cite{chen2000survey,baldauf2007survey}, е направен збир на истражувањето во
оваа област, а постојат и многу примери на апликации  и рамки за развој на
апликации со знаење за контекстот.


\section{Опис на проблемот}

Денес сме сведоци на огромен број мобилни апликации наменети за новите платформи
и најчесто наменети за групата на паметни телефони. Развивачите на овие
апликации се најчесто програмери кои поголемиот дел од своето искуство го имаат
во развој на апликации за персонални компјутери или во поново време веб
апликации во ерата на интернетот. Иако со иновациите во корисничкиот интерфејс
кој ги промовираат новите платформи за развој на мобилни апликации, корисничкото
искуство постојано се подобрува сепак најголем проблем на овие апликации
останува да се пронајде вистинскиот начин на развој на мобилна апликација која
ќе се вклопува во новата ера во која луѓето се во постојано движење и користат
многу компјутери. За некои од овие компјутери како што се паметните телефони и
се многу лични уреди, многу големо значење има вклучување на информации од
контекстот на корисникот во апликациите кои ги користи. Овие информации се пред
сè неговата локација, ситуацијата во која се наоѓа и други елементи од
контекстот кои мобилните апликации ги издигнуваат на едно повисоко ниво во однос
на апликациите за персонални компјутери и веб апликации со што ги прават многу
повеќе персонализирани.

Од аспект корисниците на мобилните уреди многу големо значење има контекстот во
кој се наоѓаат при користењето на овие уред. Контекстот на корисникот го
сочинуваат информации како локацијата, времето, опкружувањето или сите останати
елементи во неговата околина. Сите овие информации се од големо значење доколку
апликацијата знае за нив и може да го прилагоди своето извршување со што може да
понуди подобри сервиси и подобро корисничко искуство. Во овој магистерски труд е
презентиран процесот и методологијата на развој и архитектура на мобилни
апликации со користење на контекстот на корисникот. Развојот и архитектурата на
вакви апликации бараат истражување во области поврзани со собирање, обработка и
искористување на контекстот на корисникот.

Како резултат на истражувањето развиена е апликацијата наречена Гео Настани која
претставува систем за организација, споделување и промовирање на приватни и
јавни настани од различен карактер. Во оваа апликација се користи социјалниот
контекст на корисникот, односно неговите пријатели и врски, а се користи и
неговата локација но и други секундарни информации кои произлегуваат од
контекстот. Во описот на архитектурата и дизајнот направена е детална анализа на
архитектурните стилови кои може да се применат во ваков вид на софтверски
систем. Во процесот на развој направени се тестирања и евалуација на неколку
важни модули од архитектурата на системот, како што се методите за одредување на
локацијата на корисникот и начините на пренесување податоци меѓу различните
компоненти во системот.

Целта на апликацијата е да ги идентификува клучните аспекти во развојот на
современите мобилни апликации со знаење и употреба на контекстот на корисникот.
Гео Настани е софтверски систем кој помага во организирањето на настани, нивна
промоција, поканување на пријатели, препорака на интересни настани и слични
функционалности. Мобилната апликација со знаење на контекстот на корисникот
воведува многу дополнителни вредности на овој систем. Корисниците имаат постојан
пристап до системот каде и да се наоѓаат, може да пребаруваат настани во
зависност од нивната локација, да го означат своето присуство на некој настан,
како и да споделат информации во вид на кратки пораки или фотографии од самиот
настан корисни за другите корисници.

Придонесот на овој магистерски труд е во истражувањето и документирањето на
процесот на развој на современи мобилни апликации од аспект на идентификување на
најважните проблеми, предлог имплементација на нивни решенија, како и евалуација
на тие решенија. Идејата за софтвер за создавање записи и организација на
настани не е нова и е широко применета во многу системи на календари и друштвени
социјални мрежи. Но ниту една од овие апликации нема мобилна имплементација, а
уште помалку го идентификува користи или на каков било начин го вклучува
контекстот на корисникот. Затоа дел од придонесот е и самата имплементација на
апликацијата Гео Настани како прв обид за имплементација на мобилна апликација
за организација на настани со целосно вклучување на корисникот преку неговата
локација, пријатели и други контекстуални информации со што јасно се
идентификуваат придобивките и додадена вредност на целиот систем.

