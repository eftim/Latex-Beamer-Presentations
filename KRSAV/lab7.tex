\documentclass[12pt,a4paper]{exam}
\usepackage{amsmath}
\usepackage{amsfonts}
\usepackage{amssymb}
\usepackage[T2A]{fontenc}
\usepackage[utf8]{inputenc}
\usepackage{listings}
\usepackage{color}

\lstset{language=C,captionpos=b,
tabsize=4,frame=lines,
basicstyle=\ttfamily,
keywordstyle=\color{blue},
commentstyle=\color{green},
stringstyle=\color{red},
breaklines=true,showstringspaces=false}

\begin{document}
\pagestyle{headandfoot}
\header{\textbf{ФИНКИ\\Концепти за развој на
софтвер}}{}{\large{\textbf{Лабораториска вежба 7}}}
\headrule
\cfoot{Страна \thepage}
\begin{center}
\Large{\textbf{Покажувачи}}
\end{center}
\begin{questions}

\question
Во дадена низа од n природни броеви (n и низата се читаат од тастатура) соседните заемно
прости броеви (немаат заеднички делители) да си ги заменат позициите. На ист елемент од
низата може најмногу еднаш да му се изврши замена. Да се испечати изменетата низа.
Замената да се реализира во посебна функција.\\
\emph{Забелешка}: Задачата да се реши со помош на покажувачи.

\question
Да се напише функција за ротирање на елементите на дадена низа за m места (ако m е
позитивен низата се ротира во десно, а ако е негативен низата се ротира во
лево).\\ \emph{Забелешка}: Задачата да се реши со помош на покажувачи
\\\emph{На пример}:\\
\texttt{1 5 8 12 23 90 (m = 3)}\\
\texttt{12 23 90 1 5 8}

\question
Да се напише функција за сортирање на низа од цели броеви. Потоа да се напише
функција за спојување на две низи од цели броеви сортирани во растечки редослед.
Оваа функција треба во нова низа да ги смести веќе сортираните две низи во
нова низа која ќе биде исто така сортирана. Во оваа функција не смее да се
повикува функијата за сортирање.
\\\emph{Забелешка}: Задачата да се реши со помош на покажувачи. 
\\\emph{На пример}:\\ Од низите: \texttt{1 3 5 7 8} и \texttt{2 4 6} ќе се добие
низата \texttt{1 2 3 4 5 6 7 8}
\begin{lstlisting}
void sort(int *a, int n); // funkcija za sortiranje
// funkcija za spojuvanje na dvete sortirani
// nizi a i b vo nova niza c
void merge(int *a, int *b, int *c); 
\end{lstlisting}
\end{questions}
\end{document}