\usetheme{FINKI}
\usepackage{thumbpdf}
\usepackage{wasysym}
\usepackage{ucs}
\usepackage[T2A]{fontenc}
\usepackage[utf8]{inputenc}
\usepackage{pgf,pgfarrows,pgfnodes,pgfautomata,pgfheaps,pgfshade}
\usepackage{verbatim}
\usepackage{listings}
\usepackage{fancybox} 



\pdfinfo
{
  /Title       (AV2)
  /Creator     (Tomche Delev)
  /Author      (Tomche Delev)
}


\title[АВ2]{Аудиториски вежби 2}
\subtitle{Вовед во програмскиот јазик C\\ Структура на програма\\ Променливи, константи, оператори\\ Приказ на податоци}
\author{Концепти за развој на софтвер}
\date{}
\pgfdeclareimage[width=0.6\paperwidth]{finki_logo}{finki_name}
\titlegraphic{\pgfuseimage{finki_logo}}



\begin{document}

\frame[plain]{\titlepage}

%Automatic table of contents
%\section*{}
%\begin{frame}
%  \frametitle{Содржина}
%  \tableofcontents[section=1,hidesubsections]
%\end{frame}

\AtBeginSection[]
{
  \frame<handout:0>
  {
    \frametitle{Outline}
    \tableofcontents[currentsection,hideallsubsections]
  }
}

\AtBeginSubsection[]
{
  \frame<handout:0>
  {
    \frametitle{Outline}
    \tableofcontents[sectionstyle=show/hide,subsectionstyle=show/shaded/hide]
  }
}

\newcommand<>{\highlighton}[1]{%
  \alt#2{\structure{#1}}{{#1}}
}

\newcommand{\icon}[1]{\pgfimage[height=1em]{#1}}

\lstset{language=C,captionpos=b,
tabsize=4,frame=lines,
basicstyle=\scriptsize\ttfamily,
keywordstyle=\color{blue},
commentstyle=\color{lightgray},
stringstyle=\color{violet},
breaklines=true,showstringspaces=false}

%%%%%%%%%%%%%%%%%%%%%%%%%%%%%%%%%%%%%%%%%
%%%%%%%%%% Content starts here %%%%%%%%%%
%%%%%%%%%%%%%%%%%%%%%%%%%%%%%%%%%%%%%%%%%


\begin{frame}{Вовед во програмскиот јазик C}
\begin{itemize}
\item Развиен во лабораториите на Bell во периодот од 1969 од 1973 од страна на Dennis Ritchie
\item Еден од најшироко употребуваните јазици за програмирање со општа намена на сите времиња
\item Има огромно влијание во создавањето на многу други јазици за програмирање
	\begin{itemize}
	\item C++
	\item Objective C
	\item PHP
	\item Java
	\end{itemize}
\end{itemize}
\end{frame}

\begin{frame}[fragile]{Синтакса на C}

  \begin{block}{Азбуката е множество на следните дозволени симболи:}
  \begin{verbatim}
  	a-z, A-Z, 0-9 и ~!@#$%^&*()-+={}[]:;'"<>?/._  
  \end{verbatim}
  \end{block}

	\begin{alertblock}{Внимание!}
	Компајлерот разликува големи и мали букви!
	\end{alertblock}

	Од азбуката на C се формираат зборови кои може да бидат:
	\begin{enumerate}
	\item Клучни зборови
	\item Бројни и симболички константи
	\item Идентификатори
	\item Стрингови (низи од знаци)
	\item Оператори
	\end{enumerate}

\end{frame}

\begin{frame}{Синтакса на C}

  	\begin{block}{Множество на клучни зборови (32)}

  	\begin{tabular}{c c c c}
		\texttt{auto} & \texttt{double} & \texttt{int} & \texttt{struct} \\
		\texttt{break} & \texttt{else} & \texttt{long} & \texttt{switch} \\
		\texttt{case} & \texttt{enum} & \texttt{register} & \texttt{typedef} \\
		\texttt{char} & \texttt{extern} & \texttt{return} & \texttt{union} \\
		\texttt{const} & \texttt{float} & \texttt{short} & \texttt{unsigned} \\
		\texttt{continue} & \texttt{for} & \texttt{signed} & \texttt{void} \\
		\texttt{default} & \texttt{goto} & \texttt{sizeof} & \texttt{volatile} \\
		\texttt{go} & \texttt{if} & \texttt{static} & \texttt{while}
  	\end{tabular}
  	\end{block}
\end{frame}

\begin{frame}[fragile]{Структура на програма во C}

	Севкупниот изворен код кој се пишува во програмскиот јазик C е организиран во функции
	\linebreak
	
	\begin{columns}[t]
		\column{.5\textwidth}
			\begin{block}{Програма во C}
				\begin{lstlisting}
				int main() {
				    deklaracija_na_promenlivi;
				    programski_naredbi;
				}
				\end{lstlisting}
			\end{block}
		\column{.5\textwidth}
			\begin{block}{Програма во Паскал}
				\begin{lstlisting}
				Program ime_na_programata;
				var deklaracija_na_promenlivi;
				begin
				    programski_naredbi;
				end.
				\end{lstlisting}
			\end{block}
	\end{columns}

\end{frame}

\begin{frame}[shrink=10]{Функции во C}
 	\hfill
  	\begin{block}{\texttt{main}}
  		Главна фунција во C
  	\end{block}
  	
  	\begin{block}{\texttt{()}}
		Во мали загради се примаат влезните аргументи
  	\end{block}
  	
  	\begin{block}{\texttt{int}}
		Видот на податокот кој се враќа како резултат стои пред името на функцијата
  	\end{block}
  	
  	\begin{block}{\texttt{\{\}}}
		Телото на функцијата започнува со {, а завршува со }
  	\end{block}

	\begin{block}{\texttt{;}}
  	Сите наредби се одделуваат меѓусебно со \emph{;}
  	\end{block}

\end{frame}

\begin{frame}[fragile]{Употреба на коментари}
	\begin{itemize}
		\item За дополнително до објаснување или документирање на изворниот код се користат коментари
		\item Во C постојат два видови на коментари
		\begin{enumerate}
			\item коментари во еден ред
			\item коментари во повеќе редови
		\end{enumerate}	
		
	\end{itemize}
		
	\begin{columns}[t]
		\column{.5\textwidth}
			\begin{block}{1. Коментар во еден ред}
				\begin{verbatim}
				\\ komentar vo eden red
				\end{verbatim}
			\end{block}
		\column{.5\textwidth}
			\begin{block}{2. Коментар во повеќе редови}
				\begin{verbatim}
				\* Komentar 
				   vo povekje redovi */
				\end{verbatim}
			\end{block}
	\end{columns}		

\end{frame}

\begin{frame}[fragile]{Примери}
	\begin{exampleblock}{Пример 1}
		\begin{lstlisting}
			#include <stdio.h>
			// glavna funckija
			int main() {
			    /* funkcija za pecatenje na ekran */
			    printf("Dobredojdovte na FINKI!\n");
			    return 0;
			}
		\end{lstlisting}
	\end{exampleblock}
\end{frame}

\begin{frame}{Структура на програма во C (проширена)}
	\begin{columns}
		\column{.2\textwidth}
		\textbf{INCLUDE} секција
		\column{.8\textwidth} содржи \texttt{\#include} изрази за вклучување на надворешни библиотеки, односно користење веќе дефинирани надворешни функции
	\end{columns}
	\hfill
	\linebreak
	\begin{columns}
		\column{.2\textwidth}
		\textbf{DEFINE} секција
		\column{.8\textwidth} содржи \texttt{\#define} изрази за декларирање на константи и податочни типови
	\end{columns}	
	\hfill
	\linebreak
	\begin{columns}
		\column{.2\textwidth}
		...
		\column{.8\textwidth} дефинирање на глобални променливи и функции
	\end{columns}
	\hfill
	\linebreak
	\begin{columns}
		\column{.2\textwidth}
		\texttt{int main()}
		\column{.8\textwidth} главна функција
	\end{columns}
\end{frame}

\begin{frame}{Претпроцесор}
\begin{itemize}
	\item Во C преведувањето (компајлирањето) на програмите го извршуваат:
	\begin{itemize}
		\item претпроцесорот
		\item компајлерот
	\end{itemize}
	\item Претпроцесорот се управува со помош на т.н. директиви
	\begin{itemize}
		\item Секоја директива започнува со \#
	\end{itemize}
\end{itemize}
\end{frame}

\begin{frame}{Датотека со заглавја}
\begin{itemize}
	\item Една примена на претпроцесорски наредби е вклучување на „датотека со заглавија“ (анг. header file)
	\begin{itemize}
		\item Се користи за декларација на функции и променливи на одредена предефинирана библиотека
		\item Корисниците ја вклучуваат „датотеката за заглавија“ со цел да ги користат функциите и надворешните променливи
	\end{itemize}
	\item Вклучување се врши со претпроцесорската директива \texttt{\#include}
	\begin{itemize}
		\item Наредбата \texttt{\#include} предизвикува копија од даденa датотека да се вклучи на местото каде што е испишана директивата
	\end{itemize}
\end{itemize}
\end{frame}

\begin{frame}[fragile]{Форми на \texttt{include}}
\begin{itemize}
	\item Има две форми на оваа директива:
	\begin{itemize}
		\item датотеката која се вклучува може да биде ставена во наводници(""), 
		\item или во аголни загради (<>)
	\end{itemize}
	\begin{exampleblock}{Пример}
		\begin{verbatim}
		#include <imedatoteka.h>
		#include "imedatoteka.h"
		\end{verbatim}
	\end{exampleblock}
	\item Разликата е во локацијата во која препроцесорот јa бара датотеката која треба да ја вклучиме
	\begin{itemize}
		\item Со аголни загради (се користат за датотеки од стандардните библиотеки) 
		\item Со наводници (препроцесорот прво ја бара датотеката во истиот директориум каде што се наоѓа С датотеката која треба да се компајлира)
	\end{itemize}
\end{itemize}

\end{frame}

\begin{frame}{Променливи (variables)}
\begin{itemize}
\item Променливите се симболички имиња за места во меморијата во кои се чуваат некакви вредности
\item Сите променливи пред да се користат треба да се \emph{декларираат}
\item Со секое ново сместување на вредност во променливата, старата вредност се брише
\end{itemize}

Начин на декларација на променливи:
\linebreak
\begin{columns}
		\column{.32\textwidth} \fbox{Вид на променливата}
		\column{.32\textwidth} \fbox{Име на променливата}
		\column{.05\textwidth} \fbox{ = }
		\column{.26\textwidth} \fbox{Почетна вредност}
		\column{.05\textwidth} \fbox{ ; }
\end{columns}	

\end{frame}

\begin{frame}{Видови на променливи}
\Large{Видови на променливи во C}
\linebreak
\linebreak
\begin{tabular}{c|c|c}
\textbf{Цели броеви} & \textbf{Знаковни} & \textbf{Децимални}\\
\hline
\texttt{int} & \texttt{char} & \texttt{float} \\
\hline
\texttt{short} & & \texttt{double} \\
\hline
\texttt{long} & &
\end{tabular}
\end{frame}

\begin{frame}{Дефинирање на имиња на променливи}
\begin{itemize}
\item При именувањето на променливите може да се користат:
\begin{itemize}
\item мали букви од a до z;
\item големи букви од A до Z;
\item цифри од 0 до 9 (не смее да започнува со цифра);
\item знак за подвлекување \_ кој се третира како буква (не е препорачливо да започнува со \_);
\end{itemize}
\end{itemize}
\begin{alertblock}{Треба да се внимава!}
\begin{itemize}
\item најчесто должината на имињата на променливите е до 32 знаци
\item С ги разликува малите и големите букви!
\end{itemize}
\end{alertblock}
\end{frame}

\begin{frame}[fragile]{Примери}
	\begin{exampleblock}{Пример 2}
		\begin{lstlisting}
			#include <stdio.h>

			int main() {
			    int a, b, c;
			    a = 5;
			    b = 10;
			    c = a + b;
			    return 0;
			}
		\end{lstlisting}
	\end{exampleblock}
\end{frame}

\begin{frame}{Константи}
\begin{itemize}
\item Со помош на константи се означуваат вредности кои не се менуваат во текот на извршувањето на програмата
\item Секоја константа припаѓа на некој од видовите на податоци
\item Во C постојат неколку типови на константи:
\begin{itemize}
\item децимални: \texttt{1, -23, 15}
\item октални: \texttt{015, 035, 0205}
\item хексадецимални: \texttt{0x25, 0xA4C}
\item реални: \texttt{3.5F, -2.845F, 1.34e-9}
\item знаковни: \texttt{'a', '\_', 'e'}
\item текстуални: \texttt{" ", "Koncepti za razvoj na softver"}
\end{itemize}
\end{itemize}
\end{frame}

\begin{frame}{Одредување на типот на константите}
\begin{itemize}
\item Одредувањето на типот на променливите е едноставно (се гледа од самата декларација на променливата)
\item Константите не се декларираат и нивниот тип се одредува преку начинот на кој се напишани:
\begin{itemize}
\item Броевите кои содржат "." или "е" се {\color{blue}\texttt{double}}: \texttt{3.5, 1е-7, -1.29е15}
\item За наместо double да се користат {\color{blue}{\texttt{float}}} константи на крајот се додава "F": \texttt{3.5F, 1e-7F}
\item За {\color{blue}\texttt{long double}} константи се додава "L": \texttt{1.29е15L, 1e-7L}
\item Броевите без ".", "е" или "F" се {\color{blue}\texttt{int}}: \texttt{1000, -35}
\item За {\color{blue}\texttt{long int}} константи се додава "L": \texttt{9000000L}
\end{itemize}
\end{itemize}
\end{frame}

\begin{frame}[fragile]{Именувани константи (1)}
\Large{Именуваните константи се креираат со користење на клучниот збор \texttt{const}}
\begin{exampleblock}{Пример 3}
		\begin{lstlisting}
			#include <stdio.h>

			int main() {
			    const long double pi = 3.141592653590L;
			    const int denovi_vo_nedelata = 7;
			    const nedela = 0; /* po default int */
			    denovi_vo_nedelata = 1; /* greshka */
			}
		\end{lstlisting}
	\end{exampleblock}
\end{frame}

\begin{frame}[fragile]{Именувани константи (2)}
Именуваните константи може да се креираат и со користење на претпроцесорот и за нив по правило се користат големи букви
	\begin{exampleblock}{Пример 3}
		\begin{lstlisting}
			#include <stdio.h>
			#define PI 3.141592653590L
			#define DENOVI_VO_NEDELATA 7
			#define NEDELA 7		
			int main() {
			    long double pi = PI;
			    int den = NEDELA;
			}
		\end{lstlisting}
	\end{exampleblock}
\end{frame}

\begin{frame}{Оператори}
\begin{itemize}
\item Операторите се користат за градење на изрази, при што операциите се изведуваат од лево надесно со што се применува правилото на приоритет на операторите во нивното изведување
\item Постојат три видови на оператори
\begin{itemize}
\item Аритметички оператори
\item Релациони оператори
\item Логички оператори
\end{itemize}
\end{itemize}
\end{frame}

\begin{frame}{Аритметички оператори}
Се применуваат на броеви (цели или децимални)
\linebreak
\begin{center}
\begin{tabular}{c|c}
\textbf{Оператор} & \textbf{Операција}\\
\hline
\texttt{+} & Собирање \\
\texttt{-} & Одземање \\
\texttt{*} & Множење \\
\texttt{/} & Делење \\
\texttt{\%} & Делење по модул
\end{tabular}
\end{center}
\end{frame}

\begin{frame}{Релациони оператори}
Се применуваат над било кои споредливи типови на податоци, а резултатот е цел број 0 (неточно) или 1 (точно).
\begin{center}
\begin{tabular}{c|c}
\textbf{Оператор} & \textbf{Значење}\\
\hline
\texttt{<} & Помало \\
\texttt{<=} & Помало еднакво \\
\texttt{>} & поголемо \\
\texttt{>=} & поголемо еднакво \\
\texttt{==} & еднакво \\
\texttt{!=} & различно
\end{tabular}
\end{center}
\end{frame}

\begin{frame}{Логички оператори}
Се користат најчесто во комбинација со релационите оператори за формирање на сложени логички изрази, кои повторно враќаат резултат 0 или 1
\linebreak
\begin{center}
\begin{tabular}{c|c}
\textbf{Оператор} & \textbf{Операција}\\
\hline
\texttt{\&\&} & Логичко И \\
\texttt{||} & Логичко ИЛИ \\
\texttt{!} & Негација
\end{tabular}
\end{center}
\end{frame}

\begin{frame}{Дополнителни оператори}
\begin{itemize}
\item Оператор за доделување =
\item Оператори за инкрементирање и декрементирање (++, --)
\item Користење на операторите + и – на унарен начин
\begin{itemize}
\item X = + Y;
\item X = - Y;
\end{itemize}
\item Двојни оператори
\begin{itemize}
\item Комбинација од оператор за доделување и друг оператор (+=, -=, *=, /=, %=)
\end{itemize}
\end{itemize}
\end{frame}

\begin{frame}[fragile]{Оператор за доделување =}
\begin{itemize}
\item Сите изрази имаат вредност, дури и оние кои содржат =
\item Вредноста на таков израз е вредноста на изразот кој се наоѓа на десна страна
\item Затоа е можно и доделување од следниот облик:
\end{itemize}
\begin{verbatim}
    x = (y = 10) * (z = 5);
    x = y = z = 20;
\end{verbatim}
\end{frame}	

\begin{frame}[fragile,shrink=5]{Двојни оператори}
\begin{block}{Оператор +=}
\begin{verbatim}
a += 5; // a = a + 5;
a += b * c; // a = a + b * c;
\end{verbatim}
\end{block}
\begin{block}{Оператор -=}
\begin{verbatim}
a -= 3; // a = a – 3;
\end{verbatim}
\end{block}
\begin{block}{Оператор *=}
\begin{verbatim}
a *= 3; // a = a * 3;
\end{verbatim}
\end{block}
\begin{block}{Оператор /=}
\begin{verbatim}
a /= 3; // a = a / 3;
\end{verbatim}
\end{block}
\begin{block}{Оператор \%=}
\begin{verbatim}
a %= 3; // a = a % 3;
\end{verbatim}
\end{block}
\end{frame}	

\begin{frame}[fragile]{Работа со променливи и оператори}
	\begin{exampleblock}{Пример 4}
		\begin{lstlisting}
			#include <stdio.h>	
			int main() {
			    int a;
			    float p;
			    p = 1.0 / 2.0; /* p = 0.5 */
			    a = 5 / 2;     /* a = 2   */
			    p = 1 / 2 + 1 / 8; /* p = 0; */
			    p = 3.5 / 2.8; /* p = 1.25 */
			    a = p; /* a = 1 */
			    a = a + 1; /* a = 2; */
			    return 0;		
			}
		\end{lstlisting}
	\end{exampleblock}
\end{frame}

\begin{frame}[fragile]{Печатење на стандарден излез}
\begin{itemize}
\item Во C не постои наредба за печатење на екран
\item Се користи готова функција од библиотеката за стандарден влез и излез \texttt{stdio.h} (\textbf{st}andar\textbf{d} \textbf{i}nput/\textbf{o}utput)
\large{\texttt{\#include <stdio.h>}}
\item Функцијата која се употребува е:
\end{itemize}
\begin{verbatim}
int printf(kontrolna_niza, lista_na_argumenti)
\end{verbatim}
\begin{itemize}
\item Контролната низа содржи било каков текст, ознаки за форматот на печатење на аргументите предводени со \% 
или специјални знаци кои започнуваат со \textbackslash. 
\item Ознаките за форматот на печатење се одредуваат според видот на променливата чија вредноста треба да се испише.
\end{itemize}
\end{frame}

\begin{frame}{Ознаки за форматот на печатење}
\begin{scriptsize}
\begin{tabular}{|c|l|}
\hline \textbf{Ознака} & \textbf{Објаснување} \\ 
\hline \texttt{\%d} & за цели броеви \\ 
\hline \texttt{\%i} & за цели броеви \\ 
\hline \texttt{\%c} & за знаци \\ 
\hline \texttt{\%s} & за низа од знаци \\ 
\hline \texttt{\%e} & реален број во технички формат (е) \\ 
\hline \texttt{\%E} & реален број во технички формат (Е) \\ 
\hline \texttt{\%d} & реален број во децимален формат \\ 
\hline \texttt{\%f} & реален број во пократкиот од форматите \%е и \%f \\ 
\hline \texttt{\%g} & реален број во пократкиот од форматите \%Е и \%f \\  
\hline \texttt{\%u} & цел број без предзнак \\ 
\hline \texttt{\%o} & октален цел број без предзнак \\ 
\hline \texttt{\%x} & хексадецимален цел број без предзнак (мали букви) \\ 
\hline \texttt{\%X} & хексадецимален цел број без предзнак (мали букви) \\ 
\hline \texttt{\%p} & прикажува покажувач \\ 
\hline \texttt{\%n} & бројот на испишани знаци се доделува на аргументот \\ 
\hline \texttt{\%\%} & испишување на знакот \% \\ 
\hline 
\end{tabular} 
\end{scriptsize}
\end{frame}

\begin{frame}[fragile]{Примена на функцијата \texttt{printf}}
	\begin{exampleblock}{Пример 5}
		\begin{lstlisting}
		#include <stdio.h>
		int main() {
		   printf(" e zbor dolg %d bukvi.\n", printf("Makedonija"));
		   return 0;
		}
		\end{lstlisting}
	\end{exampleblock}
\end{frame}

\begin{frame}[fragile]{Задача 1}
	Да се напише програма која ќе ја пресметува вредноста на математичкиот израз:
	$ x = \frac{3}{2} + (5 - \frac{46 x 5}{12})$
	\begin{exampleblock}{Решение}
		\begin{lstlisting}
		#include <stdio.h>
		int main() {
		   float x = 3.0 / 2 + (5 - 46 * 5 / 12.0);
		   printf("x = %.2f\n", x);
		   return 0;
		}
		\end{lstlisting}
	\end{exampleblock}
\end{frame}

\begin{frame}[fragile]{Задача 2}
Да се напише програма која за зададена вредност на $ x $ (при декларација на променливата) ќе го пресмета и отпечати на екран $ x^2 $.
	\begin{exampleblock}{Решение}
		\begin{lstlisting}
		#include <stdio.h>
		int main() {
		   int x = 7;
		   printf("Brojot %d na kvadrat e: %d\n", x, x * x);
		   return 0;
		}
		\end{lstlisting}
	\end{exampleblock}
\end{frame}

\begin{frame}[fragile]{Задача 3}
Да се напише програма која за дадени страни на еден триаголник ќе ги отпечати на екран периметарот 
и квадратот од плоштината (нека се работи со \texttt{a=5, b=7.5, c=10.2}).
	\begin{exampleblock}{Решение}
		\begin{lstlisting}
		#include <stdio.h>
		int main() {
		   float a = 5;
		   float b = 7.5;
		   float c = 10.2;
		   float L = a + b + c;
		   float s = L / 2;
		   float P = s * (s - a) * (s - b) * (s - c);
		   printf("Perimetarot e: %.2f\n", L);
		   printf("Plostinata na kvadrat e: %.2f\n", P);
		   return 0;
		}
		\end{lstlisting}
	\end{exampleblock}
\end{frame}

\begin{frame}[fragile]{Задача 4}
Да се напише програма за пресметување на аритметичката средина на броевите 3, 5 и 12.
	\begin{exampleblock}{Решение}
		\begin{lstlisting}
		#include <stdio.h>
		int main() {
		   int a = 3, b = 5, c = 12;
		   float as = a + b + c / 3.0;
		   printf("Aritmetickata sredina e: %.2f\n", as);
		   return 0;
		}
		\end{lstlisting}
	\end{exampleblock}
\end{frame}

\begin{frame}[fragile]{Задача 5}
Да се напише програма која ќе ги отпечати на екран остатоците при делењето на бројот 19 со 2, 3, 5 и 8.
	\begin{exampleblock}{Решение}
		\begin{lstlisting}
		#include <stdio.h>
		int main() {
		   int a = 19;
		   printf("Ostatok pri delenje na %d so 2: %d\n", a, a % 2);
		   printf("Ostatok pri delenje na %d so 3: %d\n", a, a % 3);
		   printf("Ostatok pri delenje na %d so 5: %d\n", a, a % 5);
		   printf("Ostatok pri delenje na %d so 8: %d\n", a, a % 8);
		   return 0;
		}
		\end{lstlisting}
	\end{exampleblock}
\end{frame}

\begin{frame}{Материјали}{}
	Предавања, аудиториски вежби, соопштенија\\
	\href{http://courses.finki.ukim.mk/}{\textbf{courses.finki.ukim.mk}}
	\vfill
	Изворен код на сите примери и задачи\\
	\href{http://bitbucket.org/tdelev/finki-krs/}{\textbf{bitbucket.org/tdelev/finki-krs}}
	\vfill
	{\Huge Прашања ?}
\end{frame}

\end{document}
