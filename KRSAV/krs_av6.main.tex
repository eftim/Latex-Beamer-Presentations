
\usetheme{FINKI}
\usepackage{thumbpdf}
\usepackage{wasysym}
\usepackage{ucs}
\usepackage[T2A]{fontenc}
\usepackage[utf8]{inputenc}
\usepackage{pgf,pgfarrows,pgfnodes,pgfautomata,pgfheaps,pgfshade}
\usepackage{verbatim}
\usepackage{listings}
%\renewcommand*\ttdefault{lcmtt}

\pdfinfo
{
  /Title       (KRS)
  /Creator     (TeX)
  /Author      (Tomche Delev)
}


\title[АВ6]{Аудиториски вежби 6}
\subtitle{Вектори и матрици}
\author{Концепти за развој на софтвер}
\date{}
\pgfdeclareimage[width=0.6\paperwidth]{finki_logo}{finki_name}
\titlegraphic{\pgfuseimage{finki_logo}}



\begin{document}

\frame[plain]{\titlepage}

%Automatic table of contents
\section*{}
\begin{frame}
  \frametitle{Содржина}
  \tableofcontents[section=1,hidesubsections]
\end{frame}

\AtBeginSection[]
{
  \frame<handout:0>
  {
    \frametitle{Содржина}
    \tableofcontents[currentsection,hideallsubsections]
  }
}

\AtBeginSubsection[]
{
  \frame<handout:0>
  {
    \frametitle{Outline}
    \tableofcontents[sectionstyle=show/hide,subsectionstyle=show/shaded/hide]
  }
}

\newcommand<>{\highlighton}[1]{%
  \alt#2{\structure{#1}}{{#1}}
}

\newcommand{\icon}[1]{\pgfimage[height=1em]{#1}}

\lstset{language=C,captionpos=b,
tabsize=4,frame=lines,
basicstyle=\tiny\ttfamily,
keywordstyle=\color{blue},
%numbers=left,
%numberstyle=\tiny,
commentstyle=\color{lightgray},
stringstyle=\color{violet},
breaklines=true,showstringspaces=false}

%%%%%%%%%%%%%%%%%%%%%%%%%%%%%%%%%%%%%%%%%
%%%%%%%%%% Content starts here %%%%%%%%%%
%%%%%%%%%%%%%%%%%%%%%%%%%%%%%%%%%%%%%%%%%

\section{Вектори (еднодимензионални полиња)}

\begin{frame}[fragile]{Задачa 1}
Да се напише програма која за две низи кои се внесуваат од тастатура ќе провери
дали дали се еднакви или не. На екран да се испачати резултатот од споредбата.\\
Максимална големина на низите е 100.
\pause
\begin{exampleblock}{Решение 1 дел}
\begin{lstlisting}
#include<stdio.h>
#define MAX 100
int main() {
    int n1, n2, element, i;
    int a[MAX], b[MAX];
    printf("Golemina na prvata niza:  ");
    scanf("%d", &n1);
    printf("Golemina na vtorata niza:  ");
    scanf("%d", &n2);
    if(n1 != n2)
        printf("Nizite ne se ednakvi\n");
\end{lstlisting}
\end{exampleblock}
\end{frame}

\begin{frame}[fragile]{Задача 1}{Решение 2 дел}
\begin{exampleblock}{Решение 2 дел}
\begin{lstlisting}
    else {
        printf("Vnesi gi elementite od prvata niza: \n");
        for(i = 0; i < n1; ++i) {
            printf("a[%d] = ", i);
            scanf("%d", &a[i]);
        }
        printf("Vnesi gi elementite od vtorata niza: \n");
        for(i = 0; i < n2; ++i) {
            printf("b[%d] = ", i);            
            scanf("%d", &b[i]);
        }
        //proverka dali nizite se ednakvi:
        for(i = 0; i < n1; ++i)
            if(a[i] != b[i])
                break;
        if(i == n1)
            printf("Nizite se ednakvi \n");
        else
            printf("Nizite ne se ednakvi \n");    
    }
    return 0;
}
\end{lstlisting}
\end{exampleblock}
\end{frame}


\begin{frame}{Задачa 2}
Да се напише програма која за низа, чии што елементи се внесуваат од тестатура, ќе го пресмета збирот на парните елементи, 
збирот на непарните елементи, како и односот помеѓу бројот на парни и непарни елементи. Резултатот да се испечати на екран.
\begin{exampleblock}{Пример}
За низата:\\
\texttt{3 {\color{red}2} 7 {\color{red}6} {\color{red}2} 5 1}\\
На екран ќе се испечати: \\
\texttt{suma\_parni = 8}\\
\texttt{suma\_neparni = 16}\\
\texttt{odnos = 0.75}
\end{exampleblock}
\end{frame}

\begin{frame}[fragile]{Задачa 2}{Решение} 
\begin{exampleblock}{Решение}
\begin{lstlisting}
#include <stdio.h>
#define MAX 100
int main() {
    int i, n, a[MAX], brNep = 0, brPar = 0, sumNep = 0, sumPar = 0;
    printf("Vnesi ja goleminata na nizata: \n");
    scanf("%d", &n);
    printf("Vnesi gi elementite od nizata: \n");
    for(i = 0; i < n; ++i)
        scanf("%d", &a[i]);
    for(i = 0; i < n; ++i) {
        if(a[i] % 2) {
            brNep++;
            sumNep += a[i];
        } else {
            brPar++;
            sumPar += a[i];
        }
    }
    printf("Sumata na parni elementi: %d\nSumata na neparni elementi: %d\n", sumPar, sumNep);
    printf("Odnosot na parnite so neparnite elementi e %.2f\n", (float)brPar / brNep);
    return 0;
}
\end{lstlisting}
\end{exampleblock}
\end{frame}

\begin{frame}{Задачa 3}
Да се напише програма која ќе го пресмета скаларниот производ на два вектори со по n координати. 
Бројот на координати n, како и координатите на векторите се внесуваат од
тестатура. Резултатот да се испечати на екран.
\end{frame}

\begin{frame}[fragile]{Задача 3}{Решение} 
\begin{exampleblock}{Решение}
\begin{lstlisting}
#include<stdio.h>
#define MAX 100
int main() {
    int a[MAX], b[MAX], n, i, scalar = 0;
    printf("Vnesi ja goleminata na vektorite: ");
    scanf("%d", &n);
    printf("Vnesi gi koordinatite na prviot vector: \n");
    for(i = 0; i < n; ++i)
        scanf("%d", &a[i]);
    printf("Vnesi gi koordinatite na vtoriot vector: \n");
    for(i = 0; i < n; ++i)
        scanf("%d", &b[i]);
    for(i = 0; i < n; ++i)
        scalar += a[i] * b[i];
    printf("Scalarniot proizvod na vektorite e: %d\n", scalar);
    return 0;
}
\end{lstlisting}
\end{exampleblock}
\end{frame}

\begin{frame}{Задачa 4}
Да се напише програма која ќе провери дали дадена низа од n елементи која се
внесува од тастатура е строго растечка, строго опаѓачка или ниту строго растечка
ниту строго опаѓачка. Резултатот да се испечати на екран.
\end{frame}

\begin{frame}[fragile,shrink=10]{Задача 4}{Решение} 
\begin{exampleblock}{Решение}
\begin{lstlisting}
#include <stdio.h>
#define MAX 100
int main() {
    int n, element, a[MAX], i;
    short rastecka = 1, opagacka = 1;
    printf("Vnesi ja goleminata na nizata: \n");
    scanf("%d", &n);
    printf("Vnesi gi elementite od nizata: \n");
    for(i = 0; i < n; ++i)
        scanf("%d",&a[i]);
    for(i = 0; i< n-1; ++i) {
        if(a[i] >= a[i+1]) {
            rastecka = 0;
            break;
        }
    }    
    for(i = 0; i < n-1; ++i) {
        if(a[i] <= a[i+1]) {
            opagacka = 0;
            break;
        }
    }
    if(!opagacka && !rastecka)
        printf("Nizata ne e nitu strogo rastecka nitu strogo opagacka \n");
    else if(opagacka)
        printf("Nizata e strogo opagacka \n");
    else if(rastecka)
        printf("Nizata e strogo rastecka \n");
    return 0;
}
\end{lstlisting}
\end{exampleblock}
\end{frame}

\begin{frame}{Задачa 5}
Да се напише програма која што ќе ги избрише дупликатите од една низа. На крај,
да се испечати на екран новодобиената низа. Елементите од низата се внесуваат од тестатура.
\end{frame}

\begin{frame}[fragile]{Задача 5}{Решение} 
\begin{exampleblock}{Решение}
\begin{lstlisting}
#include <stdio.h>
#define MAX 100
int main() {
    int a[MAX], n, i, j, k, izbrisani = 0;
    printf("Vnesi ja goleminata na nizata: \n");
    scanf("%d", &n);
    printf("Vnesi gi elementite od nizata: \n");
    for(i = 0; i < n; ++i)
        scanf("%d",&a[i]);
    for(i = 0; i < n - izbrisani; ++i)
        for(j = i + 1; j < n - izbrisani; ++j)
            if(a[i] == a[j]) {
                for(k = j; k < n - 1 - izbrisani; ++k)
                    a[k] = a[k + 1];
                izbrisani++;
            }
    n -= izbrisani;
    printf("Rezultantnata niza e: \n");
    for(i = 0; i < n; ++i)
        printf("%d\t", a[i]);
    return 0;
}
\end{lstlisting}
\end{exampleblock}
\end{frame}

\section{Матрици (дводимензионални полиња)}

\begin{frame}{Задачa 6}
Да се напише програма која ќе испечати на екран дали дадена матрица е симетрична во однос на главната дијагонала. 
Димензиите и елементите на матрицата се внесуваат од тестатура.
\end{frame}

\begin{frame}[fragile]{Задача 6}{Решение} 
\begin{exampleblock}{Решение}
\begin{lstlisting}
#include <stdio.h>
#define MAX 100
int main() {
    int a[MAX][MAX], n, i, j;
    printf("Vnesi dimenzii na matricata: \n");
    scanf("%d", &n);
    printf("Vnesi gi elementite na matricata: \n");
    for(i = 0; i < n; ++i)
        for(j = 0; j < n; ++j)
            scanf("%d", &a[i][j]);
    for(i = 0; i < n - 1; ++i) {
        for(j = i + 1; j < n; ++j)
            if(a[i][j] != a[j][i])
                break;
        if(i != j) break;
    }
    if(i == j)
        printf("Matricata e simetricna vo odnos na glavnata dijagonala\n");
    else
        printf("Matricata ne e simetricna vo odnos na glavnata dijagonala\n");
    return 0;
}
\end{lstlisting}
\end{exampleblock}
\end{frame}

\begin{frame}{Материјали}{}
	Предавања, аудиториски вежби, соопштенија\\
	\href{http://courses.finki.ukim.mk/}{\textbf{courses.finki.ukim.mk}}
	\vfill
	Изворен код на сите примери и задачи\\
	\href{http://bitbucket.org/tdelev/finki-krs/}{\textbf{bitbucket.org/tdelev/finki-krs}}
	\vfill
	{\Huge Прашања ?}
\end{frame}

\end{document}
