
\usetheme{FINKI}
\usepackage{thumbpdf}
\usepackage{wasysym}
\usepackage{ucs}
\usepackage[T2A]{fontenc}
\usepackage[utf8]{inputenc}
\usepackage{pgf,pgfarrows,pgfnodes,pgfautomata,pgfheaps,pgfshade}
\usepackage{verbatim}
\usepackage{listings}
\renewcommand\mathfamilydefault{\rmdefault}
%\renewcommand*\ttdefault{lcmtt}

\pdfinfo
{
  /Title       (KRS)
  /Creator     (TeX)
  /Author      (Tomche Delev)
}


\title[АВ6]{Аудиториски вежби 7}
\subtitle{Функции}
\author{Концепти за развој на софтвер}
\date{}
\pgfdeclareimage[width=0.6\paperwidth]{finki_logo}{finki_name}
\titlegraphic{\pgfuseimage{finki_logo}}



\begin{document}

\frame[plain]{\titlepage}

%Automatic table of contents
%\section*{}
%\begin{frame}
%  \frametitle{Содржина}
%  \tableofcontents[section=1,hidesubsections]
%\end{frame}

\AtBeginSection[]
{
  \frame<handout:0>
  {
    \frametitle{Содржина}
    \tableofcontents[currentsection,hideallsubsections]
  }
}

\AtBeginSubsection[]
{
  \frame<handout:0>
  {
    \frametitle{Outline}
    \tableofcontents[sectionstyle=show/hide,subsectionstyle=show/shaded/hide]
  }
}

\newcommand<>{\highlighton}[1]{%
  \alt#2{\structure{#1}}{{#1}}
}

\newcommand{\icon}[1]{\pgfimage[height=1em]{#1}}

\lstset{language=C,captionpos=b,
tabsize=4,frame=lines,
basicstyle=\tiny\ttfamily,
keywordstyle=\color{blue},
%numbers=left,
%numberstyle=\tiny,
commentstyle=\color{lightgray},
stringstyle=\color{violet},
breaklines=true,showstringspaces=false}

%%%%%%%%%%%%%%%%%%%%%%%%%%%%%%%%%%%%%%%%%
%%%%%%%%%% Content starts here %%%%%%%%%%
%%%%%%%%%%%%%%%%%%%%%%%%%%%%%%%%%%%%%%%%%


\begin{frame}[fragile]{Дефинирање на функции}{Потсетување од предавања}
\begin{block}{Дефиниција на функции}
\begin{lstlisting}
return_type function_name(arguments_list) {
/* function body */
}
\end{lstlisting}
\end{block}
\begin{itemize}
    \item return\_type - типот на вредноста која што ја враќа функцијата
    \item function\_name - името на функцијата
    \item arguments\_list - листата со формални аргументи ги содржи аргументите
    заедно со нивните типови разделени со запирка телото на функцијата ги содржи истите елементи како и самата \texttt{main()} функција    
\end{itemize}
\end{frame}

\begin{frame}[fragile]{Повикување на функции}{Потсетување од предавања}
\begin{block}{Повикување на функции}
\begin{lstlisting}
function_name(arguments_list);
\end{lstlisting}
\end{block}
\begin{itemize}
    \item function\_name – името на веќе дефинираната  функција
    \item arguments\_list - листата на аргументи е со вистински аргументи кои
    што ако се повеќе се одделуваат со запирка
\end{itemize}
\end{frame}

\begin{frame}[fragile]{Функции од математичката библиотека}{\texttt{math.h}}
\begin{itemize}
    \item Во C постои стандардна математичка библиотека \texttt{math.h} која што содржи многу готови математички функции
    \item Може да се употребуваат доколку претходно се уклучи математичката
    библиотека \texttt{\#include<math.h>}
    \item Сите функции од стандардната библиотека \texttt{math.h} примаат аргументи од
    типот \texttt{double} и враќаат вредност од истиот тип
\end{itemize}
\begin{block}{Вклучување на математичката библиотека}
\begin{lstlisting}
#include <math.h>
\end{lstlisting}
\end{block}
\end{frame}

\begin{frame}{Најчесто користени математички функции}{\texttt{math.h}}
\begin{tabular}{l | l}
\texttt{sqrt(x)} &  квадратен корен од х\\
\hline
\texttt{exp(x)} & експоненцијална функција  ех\\
\hline
\texttt{log(x)} & природен логаритам од х (со основа е)\\
\hline
\texttt{log10(x)} & логаритам од х со основа 10 \\
\hline
\texttt{fabs(x)} & апсолутна вредност од х\\
\hline
\texttt{ceil(x)} & заокружува х на најмалиот цел број не помал од х\\
\hline
\texttt{floor(x)} & заокружува х на најголемиот цел број не поголем од х\\
\hline
\texttt{pow(x, y)}  & х на степен у\\
\hline
\texttt{fmod(x, y)}  & остаток од х/у како реален број\\
\hline
\texttt{sin(x)} & синус од х (во радијани) \\
\hline
\texttt{cos(x)} & косинус од х (во радијани)\\
\hline
\texttt{tan(x)} & тангенс од х (во радијани)
\end{tabular}

\end{frame}

\begin{frame}{Задача 1}
Да се напише програма која што ќе ги отпечати сите четирицифрени природни броеви
кои се деливи со збирот на двата броја составени од првите две цифри и од
последните две цифри на четирицифрениот број. На крајот да отпечати колку вакви
броеви се пронајдени.
\begin{exampleblock}{Пример}
\texttt{3417 е делив со 34 + 17}\\
\texttt{5265 е делив со 52 + 65}\\
\texttt{6578 е делив со 65 + 78}
\end{exampleblock}
\end{frame}

\begin{frame}[fragile]{Задача 1}{Решение}
\begin{lstlisting}
#include <stdio.h>
int zbir_po_2cifri(n) {
    return (n % 100) + (n / 100);
}
int main() {
    int i, br = 0;
    for(i = 1000; i <= 9999; i++) {
        if (i % zbir_po_2cifri(i) == 0) {
            printf("Brojot %d go zadovoluva uslovot\n",i);
                br++;
        }
    }
    printf("Pronajdeni se %d broevi koi go zadovoluvaat uslovot\n",br);
    return 0;
}
\end{lstlisting}
\end{frame}


\begin{frame}{Задачa 2}
Да се напише програма која за даден природен број ја пресметува разликата помеѓу најблискиот поголем од него прост број и самиот тој број.
\begin{exampleblock}{Пример}
Ако се внесе \texttt{573}, програмата треба да испечати\\
\texttt{577 – 573 = 4}
\end{exampleblock}
\end{frame}

\begin{frame}[fragile]{Задачa 2}{Решение} 
\begin{lstlisting}
#include <stdio.h>
int prost(int n);
int prv_pogolem_prost(int n);
int main() {
    int broj;
    printf("Vnesi broj\n");
    scanf("%d", &broj);
    int prost = prv_pogolem_prost(broj);
    printf("%d - %d = %d\n", prost, broj, prost - broj);
    return 0;
}
int prost(int n) {
    int k;
    k = 2;
    while (k * k <= n) {
        if (n % k == 0)
            return 0;
        k++;
    }
    return 1;
}
int prv_pogolem_prost(int n) {
    do
        n++;
    while (!(prost(n)));
    return n;
}
\end{lstlisting}
\end{frame}

\begin{frame}{Задачa 3}
Да се напише програма што ќе ги отпечати сите прости броеви помали од 10000 чиј
што збир на цифри е исто така прост број. На крајот да се отпечати колку вакви
броеви се пронајдени.
\begin{exampleblock}{Пример}
\texttt{23 -> 2+3=5}\\
\texttt{179 -> 1+7+9=17}\\
\texttt{9613 -> 9+6+1+3=19}
\end{exampleblock}
\end{frame}

\begin{frame}[fragile]{Задача 3}{Решение} 
\begin{columns}
    \column{.5\textwidth}
    \begin{lstlisting}
#include <stdio.h>
int e_prost(int n) {
    int i;
    if (n < 4)
        return 1;
    else if ((n % 2) == 0)
        return 0;
    else {
        i = 3;
        while (i * i <= n) {
            if (n % i == 0)
                return 0;
            i += 2;
        }
    }
    return 1;
}
int zbir_cifri(int n) {
    int zbir = 0;
    while (n > 0) {
        zbir += (n % 10);
        n /= 10;
    }
    return zbir;
}
\end{lstlisting}
    \column{.5\textwidth}
\begin{lstlisting}
    int main() {
    int br = 0, i;
    for (i = 2; i <= 9999; i++) {
        if (e_prost(i) && e_prost(zbir_cifri(i))) {
            printf("Brojot %d go zadovoluva uslovot\n", i);
            br++;
        }
    }
    printf("Pronajdeni se %d broevi koi go zadovoluvaat uslovot\n", br);
    return 0;
}
\end{lstlisting}
\end{columns}
\end{frame}

\begin{frame}{Задачa 4}
Да се напише програма што ќе ги отпечати сите парови прости броеви што се разликуваат меѓу себе за 2. 
На крај да се отпечати и нивниот број.
\begin{exampleblock}{Пример}
\texttt{11 и 13}\\
\texttt{101 и 103}\\
\texttt{617 и 619}\\
\texttt{881 и 883}
\end{exampleblock}
\end{frame}

\begin{frame}[fragile]{Задачa 4}{Решение}
\begin{lstlisting}
#include <stdio.h>
int eprost(int n);
int main () {
    int br=0,i;
    for (i=1; i<=(1000-2); i++) {
        if (eprost(i) && eprost(i+2)) {
            printf("Prostire broevi %d I %d se razlikuvaat za 2\n", i, (i+2));
            br++;
        }
    }
    printf("Pronajdeni se vkupno %d parovi prosti broevi koi go zadovoluvaat uslovot\n",br);
    return 0;
}
int eprost(int n) {
    int i;
    if (n < 4) return 1;
    else
    if ((n%2)==0) return 0;
    else {
        i=3;
        while (i*i<=n){
            if (n%i==0) return 0;
            i+=2;
        }
    }
    return 1;
} 
\end{lstlisting}
\end{frame}

\begin{frame}{Задачa 5}
Да се пресмета збирот:\\
\texttt{1!+(1+2)!+(1+2+3)!+\ldots+(1+2+...+n)!}
\\Помош:\\
\begin{itemize}
    \item Користете функција за пресметување на збирот на првите k природни
  броеви
    \item Користете функција за пресметување факториел на еден природен број k
\end{itemize}
\end{frame}

\begin{frame}[fragile]{Задача 5}{Решение}
\begin{lstlisting}
#include <stdio.h>
int factoriel(int k) {
    int i, fact_num = 1;
    for (i = 1; i <= k; i++)
        fact_num *= i;
    return fact_num;
}
int suma(int k) {
    int i, zbir = 0;
    for (i = 1; i <= k; i++)
        zbir += i;
    return zbir;
}
int main() {
    int i, n, rezultat = 0;
    printf("Vnesete eden pozitiven cel broj \n");
    if (scanf("%d", &n) && n > 0) {
        for (i = 1; i < n; i++) {
            rezultat += factoriel(suma(i));
            printf("%d! + ", suma(i));
        }
        rezultat += factoriel(suma(n));
        printf("%d! = %d", suma(n), rezultat);
    } else
        printf("Vnesena e pogresna vrednost \n");
    return 0;
}
\end{lstlisting}
\end{frame}

\begin{frame}{Задачa 6}
Да се напише функција што прима два параметра $x$ и $n$ и враќа:
\[
   f(n) = \left\{ 
  \begin{array}{l l}
    x + \frac{x^n}{n} + \frac{x^{n+2}}{n + 2} &,x\geq0\\[10px]
    - \frac{x^{n - 1}}{n - 1} + \frac{x^{n+1}}{n + 1} &,x<0
  \end{array} \right.
\]
Потоа да се состави програма што ќе ја табелира оваа функција за прочитано n во
интервал $x\in[-4, 4]$, со чекор $0.1$.
\end{frame}

\begin{frame}[fragile]{Задачa 6}{Решение}
\begin{columns}
    \column{.5\textwidth}
    \begin{lstlisting}
#include <stdio.h>
double f(float i, int j) {
    double vrednost;
    if (i > 0)
        vrednost = i + stepen(i, j) - stepen(i, j + 2);
    else
        vrednost = -stepen(i, j - 1) + stepen(i, j + 1);
    return vrednost;
}
float stepen(float i, int j) {
    int k;
    double vrednost;
    if (i == 0)
        vrednost = 0.0;
    else {
        vrednost = 1.0;
        for (k = 1; k <= j; ++k)
            vrednost *= i;
    }
    return vrednost;
}
\end{lstlisting}
    \column{.5\textwidth}
\begin{lstlisting}
int main() {
    int n;
    float x;
    printf("Vnesi broj:\n");
    scanf("%d", &n);
    if ((n >= -2) && (n <= 1))
        printf("Neodredeno.\n");
    else {
        x = -4.0;
        for (x = -4; x <= 4; x += 0.1) {
            printf("x=%3.1f, f(x)=%10.4f\n", x, f(x, n));
        }
    }
    return 0;
}
\end{lstlisting}
\end{columns}
\end{frame}



\begin{frame}{Материјали}{}
	Предавања, аудиториски вежби, соопштенија\\
	\href{http://courses.finki.ukim.mk/}{\textbf{courses.finki.ukim.mk}}
	\vfill
	Изворен код на сите примери и задачи\\
	\href{http://bitbucket.org/tdelev/finki-krs/}{\textbf{bitbucket.org/tdelev/finki-krs}}
	\vfill
	{\Huge Прашања ?}
\end{frame}

\end{document}
